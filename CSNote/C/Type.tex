\section{Type}
\subsection{进制}
常用的进制包括 \textbf{二进制、八进制、十进制、十六进制}

\begin{example}
	将 A (十进制) 转为十六进制.

	步骤一,使用 A 除 16 取其余数和商,
	步骤二,将商计入十六进制的结果中。
	步骤三,继续执行步骤一、二直至步骤一中的余数小于除数。
	步骤四,将余数计入结果。

\end{example}

\subsection{Basic Big Data}
以下是 C 语言所支持的数据类型。

\begin{table}[h]
	\textbf{type & size & interpretation & example} \\
	char & 1 byte & one ASCII character & 'f' \\
	int & 4 bytes & binary integer & 42 \\
	float & 4 bytes & floating point number & 3.14 \\
	double & 8 bytes & floating point number & 3.124124214124 \\

\end{table}

\subsection{int}
在C语言中的 int 类型有两种表示方式,1.unsigned 表示所有从0~$2^{32}$的值,2. signed 表示的一半正值一半负值。


\subsection{float}


\subsection{double}

