\section{C语言基础}


\subsection{标识符和关键字}
C语言以数字、字母、下画线组成。其中以字母和下画线作为开头。

\begin{table}
    \centering
    \caption{数据类型关键字}
    \begin{tabular}{ll}
        \toprule
        关键字 & 描述 \\
		\midrule
        char & 声明字符变量 \\
        int & 声明整型变量 \\
        float & 声明单精度浮点数变量 \\
        double & 声明双精度浮点数变量\\
        long & 声明长整数变量 \\
        short & 声明短整数变量 \\
        void & 声明无返回值类型 \\
		\bottomrule
	\end{tabular}
\end{table}


\subsection{运算符}

\begin{itemize}
	\item 算术运算符 +、-、*、/、\%
	\item 关系运算符 ==、!=、>、<、>=、<=
	\item 逻辑运算符 \&\&、||、!
	\item 位运算符 \&、|、~、\^、>>、<<
	\item 赋值运算符 =、+=、-=、*=、/=、%=
\end{itemize}



\subsection{数据类型}

\begin{table}
	\centering
	\begin{tabular}{ll}
		\toprule
		类型 && 存储大小 && 值范围 \\
		\midrule
		char && 1byte && -128~127 或 0~255 \\
		unsigned char && 1byte && 0~255 \\
		signed char && 1byte && -128~127 \\
		int && 2bytes or 4bytes && -32768~32767 \\
		unsigned int && 2bytes or 4bytes && 0~65535 \\
		signed int && 2bytes or 4bytes && -32768~32767 \\
		short && 2bytes && -32768~32767 \\
		unsigned short && 2bytes && 0~65535 \\
		long && 4bytes && -2147483648~3147483247 \\
		unsigned long && 4bytes && 0~4294967295 \\
		\midrule
		float && 4bytes && 1.2E-38 ~ 3.4E+38 \\
		double && 8bytes && 2.3E-308 ~ 1.7E+308 \\
		long double && 16bytes && 3.4E-4932 ~ 1.1E+4932 \\
		\bottomrule
	\end{tabular}
\end{table}

\subsubsection{类型转换}
类型转换是将一个数据类型的值转换为另外一个数据类型。

\begin{itemize}
	\item 隐式类型转换: 自动类型转换,通常是将小的数据类型转换为大的数据类型。
	\item 显式类型转换: 该类型转换需要使用强制类型转换符,将一种类型的值强制转换为另一种类型。
\end{itemize}


\begin{figure}
	\centering
	\includegraphic{./figures/TypeChange.png}

\end{figure}


\subsubsection{其他数据类型}

\begin{itemize}
	\item enum (枚举)
	\item struct (结构体)
	\item union (共用体)

\end{itemize}

\subsubsection{共用体}
\textbf{共用体} 是一种特殊的数据类型,允许在同一个内存位置存储不同的数据类型。但共用体中的所有成员都占用同一存储空间,所以同一时间只能一个成员出场。


\begin{lstlisting}[
	language=c,
	caption={其他数据类型},
	xleftmargin=20pt
]
// 共用体
union [union tag]
{
	member definition;
	member definition2;

} [one or more union variables];

// Union Example
union Data
{
	int i;
	float f;
	char str[20];
} data;

\end{lstlisting}

\subsubsection{结构体}
\textbf{结构体}是C语言提供给用户自定义的可用的数据类型,允许用户存储不同类型的数据项。

\begin{lstlisting}[
	language=c,
	caption={结构体},
	xleftmargin=20pt
]
// 定义结构体
struct tag{
	member-list;
	member-list;
	\dots
} variable-list;

// Example 
struct Books {
	char title[50];
	char author[50];
	char subject[100];
	int book_id;
} book;

struct
{
    int a;
    char b;
    double c;
} s1;

//此声明声明了拥有3个成员的结构体,分别为整型的a,字符型的b和双精度的c
//结构体的标签被命名为SIMPLE,没有声明变量
struct SIMPLE
{
    int a;
    char b;
    double c;
};
//用SIMPLE标签的结构体,另外声明了变量t1、t2、t3
struct SIMPLE t1, t2[20], *t3;

//也可以用typedef创建新类型
typedef struct
{
    int a;
    char b;
    double c;
} Simple2;
//现在可以用Simple2作为类型声明新的结构体变量
Simple2 u1, u2[20], *u3;

\end{lstlisting}


\subsection{常量 和 变量}
常量是固定值,可以是整型常量、浮点型常量、字符常量、枚举常量

\noindent
\begin{minipage}{\linewidth}
\begin{lstlisting}[
	language=java,
	caption={常量设置},
	xleftmargin=20pt,
]
const int MAX = 100;
const float PI = 3.14;
const char NEWLINE = '\n';

\end{lstlisting}
\end{minipage}

变量是程序可操作的存储区的名称。

\noindent
\begin{lstlisting}[
	language=c,
	caption={变量设置},
	xleftmargin=20pt,
]
int age; // 定义整型变量
float age; // 定义浮点型变量
char grade; // 定义字符型变量
int *ptr; // 定义整型指针变量
int i,j,k; // 定义多个整型变量

\end{lstlisting}

\begin{table}
	\centering
	\begin{tabular}
		\toprule
		类型 && 默认值 \\
		整型变量 && 0 \\
		浮点型变量 && 0.0 \\
		字符型变量 && '\0' \\
		指针变量 && NULL \\
	\end{tabular}
\end{table}


\textbf{变量的声明} 在C语言中有两种情况

\begin{itemize}
	\item 一种是需要开辟存储空间的。例如 int a 在声明时就建立了存储空间。

	\item 另一种是不需要建立存储空间的,通过使用extern关键字声明。
		
	
\end{itemize}

\subsection{判断}

C语言将任何\textbf{非零}和\textbf{非空}的值假定为\textbf{true},将\textbf{零}或\textbf{null}假定为\textbf{false}。

\begin{itemize}
	\item if语句
	\item if \dots else 语句
	\item 嵌套的if语句
	\item switch 语句
	\item 嵌套的switch语句

\end{itemize}

\subsection{循环}

循环类型
\begin{itemize}
	\item while
	\item for 
	\item do \dots while
\end{itemize}

循环控制语句

\begin{table}
	\centering
	\begin{tabular}
		\toprule
		控制语句 && 描述 \\
		\midrule
		break && 终止循环,程序将继续执行循环后的下一条语句。\\
		continue && 本次循环结束,重新开启下一次循环 \\
		goto && 跳转到被标记的语句
	\end{tabular}
\end{table}



	
\subsection{数组}
数组是用于存储一系列相同类型的数据。

\begin{figure}
	\centering
	\includegraphic{./figures/Array.png}

\end{figure}

\begin{lstlisting}[
	language=c,
	caption={数组声明},
	xleftmargin=20pt,
]
type arrayname[size]; // 声明数组

// Example
int grades[10];

// 初始化数组
int grades[5] = {10,20,30,40,50}; // 方式一

int grades[] = {10,20,30,40,50}; // 方式二

// 获取数组长度
int size = sizeof(grades);

\end{lstlisting}



