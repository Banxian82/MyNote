% \iffalse meta-comment
%
% Copyright (C) 2019-2021 by Southern University of Science and Technology Computer Research Association <cra@sustech.edu.cn>
%
%
% This work may be distributed and/or modified under the
% conditions of the LaTeX Project Public License, either version 1.3
% of this license or (at your option) any later version.
% The latest version of this license is in
%   http://www.latex-project.org/lppl.txt
% and version 1.3 or later is part of all distributions of LaTeX
% version 2005/12/01 or later.
%
% This work has the LPPL maintenance status `maintained'.
%
% \fi
%
% \iffalse
%<*driver>
\ProvidersFile{neu.dtx}{2025/6/01 0.0.1 North University of China Thesis Template}
\documentclass{ltxdoc}
\usepackage{dtx-style}
\EnableCrossrefs
\CodelineIndex
\begin{document}
    \DocInput{\jobname.dtx}
\end{document}
%</driver>
% \fi
%
% \DoNotIndex{\newenvironment,\@bsphack,\@empty,\@esphack,\sfcode}
% \DoNotIndex{\addtocounter,\label,\let,\linewidth,\newcount}
% \DoNotIndex{\noindent,\normalfont,\par,\parskip,\phantomsection}
% \DoNotIndex{\providecommand,\ProvidesPackage,\refstepcounter}
% \DoNotIndex{\RequirePackage,\setcounter,\setlength,\string,\strut}
% \DoNotIndex{\textbackslash,\texttt,\ttfamily,\usepackage}
% \DoNotIndex{\begin,\end,\begingroup,\endgroup,\par,\\}
% \DoNotIndex{\if,\ifx,\ifdim,\ifnum,\ifcase,\else,\or,\fi}
% \DoNotIndex{\let,\def,\xdef,\edef,\newcommand,\renewcommand}
% \DoNotIndex{\expandafter,\csname,\endcsname,\relax,\protect}
% \DoNotIndex{\Huge,\huge,\LARGE,\Large,\large,\normalsize}
% \DoNotIndex{\small,\footnotesize,\scriptsize,\tiny}
% \DoNotIndex{\normalfont,\bfseries,\slshape,\sffamily,\interlinepenalty}
% \DoNotIndex{\textbf,\textit,\textsf,\textsc}
% \DoNotIndex{\hfil,\par,\hskip,\vskip,\vspace,\quad}
% \DoNotIndex{\centering,\raggedright,\ref}
% \DoNotIndex{\c@secnumdepth,\@startsection,\@setfontsize}
% \DoNotIndex{\,\@plus,\@minus,\p@,\z@,\@m,\@M,\@ne,\m@ne}
% \DoNotIndex{\@@par,\DeclareOperation,\RequirePackage,\LoadClass}
% \DoNotIndex{\AtBeginDocument,\AtEndDocument}
% 
% \GetFileInfo{\jobname.dtx}
% 
% 
% \def\indexname{索引}
% \IndexPrologue{\section{\indexname}}
% 
% \title{\bfseries\color{violet}\nucthesis :中北大学学位论文模板}
% \author{{\fangsong Norbury}\\{\fangsong 南方科技大学计算机研究协会 CRA}\\{\fangsong 清华大学 TUNA}\\[5pt]\texttt{NorburyMJ@outlook.com}}
% \date{v\fileversion\ (\filedate)}
% \maketitle\thispagestyle{empty}
% 
% 
% \begin{abstract}\noindent
%    此宏包旨在建立一个简单易用的中北大学学位论文模板。目前包括硕士论文。
% \end{abstract}
% 
% \vskip2cm
% \def\abstractname{免责声明}
% \begin{abstract}
% \noindent  
% \begin{enumerate}
% \item 本模板的发布遵守 \href{https://www.latex-project.org/lppl/lppl-1-3c.txt}{\LaTex{}Project Public License (1.3.c)}, 使用前请认真阅读协议内容。
% \item 本模板为作者根据中北大学研究生院发布的\href{https://grs.nuc.edu.cn/info/1054/4413.htm}{《中北大学关于撰写研究生学位论文的统一要求(新)》(以下或改称为写作指南》)}
% 编写而成,旨在供中北大学毕业生撰写学位论文使用,如有冲突请以官网规定为准。
% \item 任何个人或组织以本模板为基础进行修改、扩展而生成的新的专用模板,请严格遵守 \LaTex{} Project Public License 协议。 由于违反协议而引起的任何纠纷争端均与本模板作者无关。
\end{enumerate}
% \end{abstract}
% 
% 
% \clearpage
% \pagestyle{fancy}
% \begin{multicols}{2}[
%   \setlength{\columnseprule}{.4pt}
%   \setlength{\columnsep}{18pt}]
% \tableofcontents
% \end{multicols}
% \clearpage
% 
% \section{模板介绍}
% \nucthesis{} (\textbf{N}orth \textbf{U}niversity of \textbf{China} \LaTeX{} \textbf{Thesis} Template)
% 是为了帮助中北大学毕业生撰写学位论文而编写的 \LaTeX{} 论文模板。
% 
% 本文档将尽量完整的介绍模板的使用方法,如有不清楚之处,或者想提出改进建议,
% 可以在 \href{URL}{text}
% 
% \end{enumerate}
% 
% \section{贡献者}
% \label{sec:contributors}
%
% \nucthesis{} 的开发过程中,参与的维护者包括:
% 
% \begin{itemize}
% \item Norbury
%
%\end{itemize} 
% 
% \nucthesis{} 由 \thuthesis{}和 \sustechthesis{} 适配,感谢清华大学 TUNA 协会为模版易用性所做出的努力,感谢南方科技大学计算机协会。
% \nucthesis{} 的持续发展,离不开你们的帮助与支持。
% 
% \section{安装}
% \label{sec:installation}
% 
% 建议从下列途径下载最新发布版:
% \begin{description}
%   \item[GitHub] \url{}从 Release 下载 zip 版本。 
% \end{description}
% 
% % 模板支持在 TeX Live、MacTeX 本地编译器和 Overleaf 在线平台下进行编译,但要求 2019 年或更新的发行版。
% 当然,尽可能使用最新的版本可以避免 bug。
% 
% \subsection{模板的组成}
% 下表列出了 \nucthesis{} 的主要文件及其功能介绍:
% 
% \begin{longtable}{l|p{8cm}} 
% \toprule
% {\heiti 文件(夹)} & {\heiti 功能描述}\\\midrule
% \endfirsthead
% \midrule
% {\heiti 文件(夹)} & {\heiti 功能描述}\\\midrule
% \endhead
% \endfoot
% \endlastfoot
% nucthesis.ins & \textsc{DocStrip} 驱动文件(开发用) \\
% nucthesis.dtx & \textsc{DocStrip} 源文件(开发用) \\\midrule
% nucthesis.cls & 模板类文件 \\
% nucthesis-example.tex & 示例文档主文件 \\
% ref/ & 示例文档参考文献目录 \\
% data/ & 示例文档章节具体内容 \\
% figures/ & 示例文档图片路径 \\
% nuc-setup.tex & 示例文档基本配置 \\\midrule
% Makefile & Makefile \\
% latexmkrc & latexmk 配置文件 \\
% README.md & Readme \\
% \textbf{nucthesis.pdf} & 用户手册(本文档)\\\bottomrule
% \end{longtable}
% 
% 几点说明:
% \begin{itemize}
%   \item 使用前请阅读文档:\file{nucthesis.pdf}
% \end{itemize}
% 
% \subsection{生成模板}
% \label{sec:generate-cls}
% 模板的源文件 (\file{nucthesis.dtx} 中包含了大量注释,需要将这些注释去除生成轻量级的 \file{.cls} 文件供 \cs{documentclass} 调用。
% 
% \begin{shell}
%   $ xetex nucthesis.ins $
% \end{shell}
% 
% 
% \subsection{编译论文}
% \label{sec:generate-thesis}
% 本节介绍几种常见的生成论文的方法。用户可以根据自己的情况选择。
% 
% 在撰写论文时,我们 \textbf{不推荐} 使用原有的 \file{nucthesis-example.tex} 这一名称。
% 建议复制一份,改作他名(如 \file(thesis.tex) 或者 \file{main.tex} )。
% 需要注意,如果使用了来自 \file{data} 目录中的 \file{tex} 文件,
% 则重命名文件后,其顶端的 \texttt{!TeX root} 
% 
% \subsubsection{GNU make}
% \label{sec:make}
% 如果用户可以使用 GNU make 工具,这是最方便的办法。
% 所以 \sustechthesis{} 提供了 \file{Makefile}:
% \begin{shell}
%   $ make thesis $   # 生成论文示例 sustechthesis-example.pdf 
%   $ make doc  $     # 生成说明文档 sustechthesis.pdf
%   $ make clean   $  # 清理编译生成的辅助文件
% \end{shell}
% \note[注意:]{Makefile 还提供其他命令,可以自行查看。}
%
% 需要注意,如果更改了主文件的名称,则需要修改 \file{Makefile} 顶端的 \texttt{THESIS} 变量定义。
%
% \subsubsection{latexmk}
% \label{sec:latexmk}
% \texttt{latexmk} 命令支持全自动生成 \LaTeX{} 编写的文档,并且支持使用不同的工具
% 链来进行生成,它会自动运行多次工具直到交叉引用都被解决。建议没有配置 GNU make 工具的 Windows 用户
% 采用该种方式。
% \begin{shell}
%   $ latexmk sustechthesis-example.tex  # 生成示例论文 sustechthesis-example.pdf
%   $ latexmk sustechthesis.dtx          # 生成说明文档 sustechthesis.pdf
%   $ latexmk -c                         # 清理编译生成的辅助文件
% \end{shell}
% \texttt{latexmk} 的编译过程是通过 \file{latexmkrc} 文件来配置的,如果要进一步了解,
% 可以参考 \pkg{latexmk} 文档。
%
% \subsubsection{\XeLaTeX}
% \label{sec:xelatex}
% 如果用户无法使用以上两种较为方便的编译方法,就只能按照以下复杂的办法手动编译。
%
% 首先,更新模板:
% \begin{shell}
%   $ xetex sustechthesis.ins                       # 生成 sustechthesis.cls
% \end{shell}
%
% 然后,生成论文:
% \begin{shell}
%   $ xelatex sustechthesis-example.tex
%   $ bibtex sustechthesis-example.aux              # 生成 bbl 文件
%   $ xelatex sustechthesis-example.tex             # 解决引用
%   $ xelatex sustechthesis-example.tex             # 生成论文 PDF
% \end{shell}
%
% 下面的命令用来生成用户手册:
% \begin{shell}
%   $ xelatex -shell-escape sustechthesis.dtx
%   $ makeindex -s gind.ist -o sustechthesis.ind sustechthesis.idx
%   $ xelatex -shell-escape sustechthesis.dtx
%   $ xelatex -shell-escape sustechthesis.dtx  # 生成说明文档 sustechthesis.pdf
% \end{shell}
% 
% 
% \subsection{升级}
% \label{sec:upgrade}
% 如果需要升级 \nucthesis{},应当从 GitHub 下载最新的版本,
% 将 \file{nucthesis.dtx} 和 \file{nucthesis.ins} 拷贝至工作目录覆盖相应的文件,然后按照第~\ref{sec:generate-cls} 节的内容生成性的模板和使用说明。
% 
% 有时模板可能进行了重要的修改,不兼容已写好的正文内容,用户应按照示例
% 文档重新调整。
% 
% \section{使用说明}
% \label{sec:usage}
% 本手册假定用户已经能处理一般的 \LaTeX{} 文档,并对 \hologo{BibTeX} 有一定了解。如果
% 从未接触过 \TeX{} 和 \LaTeX,建议先学习相关的基础知识
% 
% \subsection{示例文件}
% \label{sec:useguide}
% 
% 模板的核心文件有:\file{nucthesis.cls},
% 但如果没有示例文档会较难下手,所以推荐从模板自带的示例文档入手,
% 其中包括了论文写作所用到的所有命令及其使用方法,只需用自己的内容进行相应的替换即可。
% 对于不清楚的命令可查询本手册。下面的例子描述了模板中章节的组织形式,具体内容可以参考模板附带的 \file{nucthesis-example.tex} 、\file{nucthesis-setup.tex} 和 \file{data/}
% 
% \subsection{论文选项}
% \label{sec:option}
% 绝大部分论文选项的设置位于 \file{nuc-setup.tex} 文件,对应选项附近也有相关注释。
% 其他论文选项设置位于 \file{nucthesis-example.tex}。
% 
% 本节中的 \emph{key-value} 选项只能在文档类 \file{nucthesis-example.tex} 的选项中进行设置,
% 不能用于 \file{nuc-setup.tex} 中的 \cs{nucsetup} 命令。
% 
% \subsubsection{学位}
% \DescribeOption{degree}
% 选择学位,可选:
% \option{master} (默认),\option{doctor}。
% 
% \begin{latex}
%   % 硕士论文
%   \documentclass[degree=master]{nucthesis}
% \end{latex}
% 
% \subsubsection{书写语言}
% \DescribeOption{language}
% 定义论文的主要语言,如标题章节等。
% 在正文中设置 \option{language} 只修改接下来部分的书写语言,
% 如标点格式,图标名称,但不影响章节标题等。
% 
% \begin{latex}
%   % 英文为主要语言
%   \documentclass[language=english]{nucthesis}
% \end{latex}
% 
% 论文的一些部分(如英文摘要)要求使用特定的语言,
% 模板已经进行配置,并在这些部分结束后自动恢复为主要语言。
% 
% 注意,\textbf{用户须提前与导师和院系的审查教师确认使用何种语言书写论文}。
% 例如:部分院系允许(或要求)外籍导师的学生采用英文书写论文,然而中国籍导师的学生仅能使用中文。
% 
% 
% \subsection{字体配置}
% \label{sec:font-config}
% 模板默认可以自动检测操作系统,并配置该平台上合适的字体。
% 具体的配置策略如表~\ref{tab:font}。
% 
% \begin{table}[htb]
%   \centering
%   \caption{\nucthesis{} 自动配置字体策略}
%   \label{tab:font}
%   \begin{tabular}{ccc}
%       \toprule
%       Windows & maxOS & 其他 \\
%       \midrule
%       Times New Roman & Times New Roman & TeX Gyre Termes \\
%       Arial & Arial & TeX Gyre Heros \\
%       Courier & Courier & TeX Gyre Cursor \\
%       中易宋体 & 华文宋体 & Fandol 宋体 \\
%       中易黑体 & 华文黑体 & Fandol 黑体 \\
%       \bottomrule
%    \end{tabular}
% \end{table}
% 
% 然而自动配置的字体只能保证编译通过,但是还存在一些问题:
% \begin{enumerate}
%   \item 在其他平台上配置的TeX Gyre 系列字体,虽然在风格上比较接近 Times 和 Arial,
% 但是毕竟和《写作指南》要求的字体不一致;
%   \item Fandol 字库的字形较少,常常出现缺字的情况;
% 
% \end{enumerate}
% 
% 所以建议在提交最终版前使用 Windows 平台的字体进行编译。
%
% 用户也可以在调用 \nucthesis{} 时手动指定使用系统自带的字库,如:
% 
% \begin{latex}
%   \documentclass[fontset=windows]{nucthesis}
% \end{latex}
% 
% 允许的选项有 \option{windows}、\option{mac}、\option{fandol},详见 \pkg{ctex}、\pkg{xeCJK}、\pkg{fontspec} 等宏包的使用说明。
% 
% 如在非 Windows 系统下还想使用其字体,可使用 \option{external} 选项调用包内带的 Windows 字体:
%
% \begin{latex}
%   \documentclass[cjk-font=external]{sustechthesis}
% \end{latex}
% 
% 允许的选项有 \option{windows}、\option{mac}、\option{fandol}、\option{external},等。
% 
% \subsection{论文设置}
% 
% 
% \clearpage
% 
% \section{实现细节}
% 
% \subsection{基本信息}
%   \begin{macrocode}
%<cls>\NeedsTeXFormat{LaTeX2e}[2017/04/15]
%<cls>\ProvidesClass{nucthesis}
%<cls>[2025/06/01 0.0.1 North University of China Thesis Template] 
% 
%   
%   \end{macrocode}
% 报错
%   \begin{macrocode}
\newcommand\nuc@error[1]{%
    \ClassError{nucthesis}{#1}{}%
}
\newcommand\nuc@warning[1]{%
    \ClassWarning{nucthesis}{#1}%
}
\newcommand\nuc@patch@error[1]{%
    \nuc@error{Failed to patch command \protect#1}%
}
\newcommand\nuc@deprecate[2]{%
    \def\nuc@@tmp{#2}%
    \nuc@warning{%
        The #1 is deprecated%
        \ifx\nuc@@tmp\@empty\else
        . Use #2 instead%
        \fi 
    }%
}

%   \end{macrocode}
% 
% 检查 \LaTeXe{} kernel 版本
% 使用了 bibtex-gbt7714 2.0 版本的接口,故要求至少 TeX Live 2020
% 
% 
%   \begin{macrocode}
\@ifl@t@r\fmtversion{2020/01/01}{}{
    \nuc@error{%
        TeX Live 2020 or later version 
        is required to compile this document%
    }
}

%   \end{macrocode}
% 
% 检查编译引擎,要求使用 \XeLaTeX。
%   \begin{macrocode}
\RequirePackage{iftex}
\ifXeTeX\else
    \nuc@error{XeLaTeX is required to compile this document}
\fi
%   \end{macrocode}
% 
% \subsection{定义选项}
% \label{sec:defoption}
% 定义论文类型以及是否涉密
%   \begin{macrocode} 
%<*cls>
\hyphenation{Thu-Thesis
SUSTech-Thesis
NUC-Thesis}
\def\thuthesis{ThuThesis}
\def\sustechthesis{SUSTechThesis}
\def\nucthesis{NUCThesis}
\def\version{0.0.1}
\RequirePackage{kvdefinekeys}
\RequirePackage{kvsetkeys}
\RequirePackage{kvoptions}
\SetupKeyvalOptions{
    family=nuc,
    prefix=nuc@,
    setkeys=\kvsetkeys
}
%   \end{macrocode}
% 
%   \begin{macro}{\nucsetup}
% 提供一个 \cs{nucsetup} 命令支持 \emph{key-value} 的方式来设置。
%   \begin{macrocode} 
\newcommand\nucsetup[1]{%
    \kvsetkeys{nuc}{#1}%
}
%   \end{macrocode}
% 
% 同时用 \emph{key-value} 的方式来定义这些接口:
%   \begin{latex} 
%       \nuc@define@key{
%           <key> = {
%               name = <name>,
%               choices = {
%                   <choice1>,
%                   <choice2>,
%               },
%               default = <default>,
%           },
%       }
%    \end{latex}
%    
% 其中 |choices| 设置允许使用的值,默认为第一个 (或者 \meta{default});
% \meta{code} 是相应的内容被设置时执行的代码。
% 
%   \begin{macrocode} 
\newcommand\nuc@define@key[1]{%
    \kvsetkeys{nuc@key}{#1}%
} 
\kv@set@family@handler{nuc@key}{%

} 
%   \end{macrocode}
% 
% \cs{nucsetup} 会将 \meta{value} 存到 \cs{nuc@\meta{key}},
% 但是宏的名字包含 “-” 这样的特殊字符时不方便直接调用,比如 |key = math-style|,
% 这是可以用 |name| 设置 \meta{key} 的别称,比如 |key = math@style|,
% 这样就可以通过 \cs{nuc@math@style} 来引用。
% |default| 是定义该 \meta{key} 时默认的值,缺省为空。
% 
% \begin{macrocode} 
    \@namedef{nuc@#1@@name}
    \def\nuc@@default{}
    \def\nuc@@choices{}
    \kv@define@key{nuc@value}{name}{
        \@namedef{nuc@#1@@name}{##1}
    }
% \end{macrocode}
% 
% 由于在定义接口时,\cs{nuc@\meta{key}@@code} 不一定有定义,
% 而且在文档类/宏包中还有可能对该 |key| 的 |code| 进行添加。
% 所以 \cs{nuc@\meta{key}@@code} 会检查如果在定义文档类/宏包时则推迟执行,否则立即执行。
% 
% \begin{macrocode} 
    \@namedef{nuc@#1@@check}{}
    \@namedef{nuc@#1@@code}{}
% \end{macrocode}
% 
% 保存下 |choices = {}| 定义的内容,在定义 \cs{nuc@\meta{name}} 后再执行。
% 
% \begin{macrocode} 
    \kv@define@key{nuc@value}{choices}{}
    \def\nuc@@choices{##1}
    \@namedef{nuc@#1@@reset}{}
% \end{macrocode}
% 
% \cs{nuc@\meta{key}@check} 检查 |value| 是否有效,并设置 \cs{ifnuc@\meta{name}@\meta{value}}。


    % Norbury注:\begin{macrocode} \end{macrocode} 在`.dtx`文件中用于包裹宏代码,既用于生成 `.sty/.cls` 文件,也用于文档展示。

% \begin{macrocode}

    % Norbury注:\@namedef{abc} 用于定义宏,且宏的名称为name。其作用相当于 \def\abc 

	\@namedef{nuc@#1@@check}{%

		\@ifundefined{%

			ifnuc@\@nameuse{nuc@#1@@name}@\@nameuse{nuc@\@nameuse{nuc@#1@@name}}%
		}{%
			\nuc@error{Invalid value "#1 = \@nameuse{nuc@#1@@name}"}%
		}%
		\@nameuse{nuc@#1@@reset}%
		\@nameuse{nuc@\@nameuse{nuc@#1name}@\@nameuse{nuc@\@nameuse{nuc@#1@@name}}true}%

	}

	% Norbury注:\kv@define@key 用于解析和设置键值对的宏命令。
	\kv@define@key{nuc@value}{default}{
		\def\nuc@@default{##1}
	}

	\kvsetkeys{nuc@value}{#2}
	\@namedef{nuc@\@nameuse{nuc@#1@@name}}{}

% \end{macrocode}

	% 第一个\meta{choice} 设为 \meta{default},并且对每个 \meta{choice} 定义 \cs{ifnuc@\meta{name}@\meta{choice}}

% \begin{macrocode}
	% Norbury注: \kv@set@family@handler 来源于kvsetkeys包中,用于设置某个“键值”族。
	\kv@set@family@handler{nuc@choice}{
		\ifx\nuc@@default\@empty 
		\def\nuc@@default{##1}
		\fi 
		\expandafter\newif\csname ifnuc@\@nameuse{nuc@#1@@name}@##1\endcsnaem
		\expandafter\g@addto@macro\csname nuc@#1@@reset\endcsname{
			\@nameuse{nuc@\@nameuse{nuc@#1@@name}@##1false}
		}
	}
	\kvsetkeys@expandafter{nuc@choice}{\nuc@@choices}
% \end{macrocode}
%
% 将 \meta{default} 赋值到 \cs{nuc@\meta{name}}, 如果非空,则执行相应的代码

% \begin{macrocode}
	\expandafter\let\csname nuc@\@nameuse{nuc@#1@@name}\endcsname\nuc@@default
	\expandafter\ifx\csname nuc@\@nameuse{nuc@#1@@name}\endcsname\@empty\else 
	\@nameuse{nuc@#1@@check}
	\fi 

% \end{macrocode}

% 定义 \cs{nucsetup} 接口。
% \begin{macrocode}
	\kv@define@key{nuc}{#1}{
		\@namedef{nuc@\@nameuse{nuc@#1@@name}}{##1}
		\@nameuse{nuc@#1@@check}
		\@nameuse{nuc@#1@@code}
	}
	
% \end{macrocode}

% 定义接口向 |key| 添加 |code| 

% \begin{macrocode}
	\newcommand\nuc@option@hook[2]{
		\expandafter\g@addto@macro\csname nuc@#1@@code\endcsname{#2}
	}
% \end{macrocode}

% \begin{macrocode}
	\nuc@define@key{
		degree = {
			choices = {
				bachelor, % 学士
				master, % 硕士
				doctor, % 博士
				postdoc, % 博士后
			},
			default = master,
		},
		degree-type = {
			default = academic,
			choices = {
				academic,
				professional,
			},
			name = degree@type,
		},
	}
% \end{macrocode}

% 论文的主要语言
% \begin{macrocode}
	main-language = {
		name = main@language,
		choices = {
			chinese,
			english,
		},
	},
% \end{macrocode}

% 用于设置局部语言。
% \begin{macrocode}
	language = {
		choices = {
			chinese,
			english,
		},
	},
% \end{macrocode}

% 字体
% \begin{macrocode}
	fontset = {
		choices = {
			windows,
			mac,
			ubuntu,
			fandol,
			none,
		},
		default = none,

	},
	system = {
		choices = {
			mac,
			unix,
			windows,
			auto,
		},
		default = auto,
	},

	font = {
		choices = {
			times,
			termes,
			auto,
			none,
			external,
		},
		default = auto,
	},
	cjk-font = {
		name = cjk@font,
		choices = {
			windows,
			mac,
			noto,
			fandol,
			auto,
			none,
			external,
		},
		default = auto,
	},
	math-font = {
		name = math@font,
		choices = {
			cambria,
			xits,
			stix,
			none,
			times,
		},
		default = cambria,
	},
	math-style = {
		name = math@style,
		choices = {
			GB,
			TeX,
		},
	},

% \end{macrocode}

% 选择打印版还是用于上传的电子版
% \begin{macrocode}
	output = {
		choices = {
			print,
			electronic,
		},
		default = print,
	},

	\newif\ifnuc@degree@graduate
	\newcommand\nuc@set@graduate{
		\nuc@degree@graduatefalse
		\ifnuc@degree@doctor
		    \nuc@degree@graduatetrue
		\fi 
		\ifnuc@degree@master
		    \nuc@degree@graduatetrue
		\fi 
	}
	\nuc@set@graduate
	\nuc@option@hook{degree}{
		\nuc@set@graduate
	}
% \end{macrocode}

% 设置默认 \option{openany}

% \begin{macrocode}
	\DeclareBoolOption[false]{openright}
	\DeclareComplementaryOption{openany}{openright}
% \end{macrocode}

% \option{raggedbottom} 选项(默认开启)
% \begin{macrocode}
	\DeclareBoolOption[true]{raggedbottom}
% \end{macrocode}

% 将选项传递给 \pkg{ctexbook}
% \begin{macrocode}
	\DeclareDefaultOption{\PassOptionsToClass{\CurrentOption}{ctexbook}}

% \end{macrocode}

% 解析用户传递的选项,并加载 \pkg{ctexbook}
	
% \begin{macrocode}
	\ProcessKeyvalOptions* 
% \end{macrocode}
%
% 设置默认 \option{openany}
% \begin{macrocode} 
	\ifnuc@openright
	\PassOptionsToClass{openright}{book}
	\else
	\PassOptionsToClass{openany}{book}
	\fi
% \end{macrocode}
% 
% 使用 \pkg{ctexbook} 类,优于调用 \pkg{ctex} 宏包。
% 
% \begin{macrocode} 
	\PassOptionsToPackage{quiet}{fontspec}
	\LoadClass[a4paper,UTF8,zihao=-4,scheme=plain,fontset=none]{ctexbook}[2017/04/01]
% \end{macrocode}
% 
%  
% \subsection{装载宏包}
% \label{sec:loadpackage}
% 
% 引用的宏包和相应的定义。
% \begin{macrocode} 
	\RequirePackage{etoolbox} % etoolbox 宏包用于定义编号。
	\RequirePackage{filehook} % filehook 宏包用于定位error
	\RequirePackage{xparse}
% \end{macrocode}
% 
% 利用 \pkg{textcase} 英文大小写转换。
% \begin{macrocode}
	\RequirePackage{textcase}
% \end{macrocode}
% 
% \begin{macrocode} 
	\RequirePackage{geometry}
% \end{macrocode}
% 
% 利用 \pkg{fancyhdr} 设置页眉页脚
% \begin{macrocode} 
	\RequirePackage{fancyhdr}
% \end{macrocode}
% 
% 利用 \pkg{titletoc} 设置目录
% \begin{macrocode}
	\RequirePackage{titletoc}
% \end{macrocode}
% 
% 利用 \pkg{notoccite} 避免目录中引用编号混乱。 
% \begin{macrocode} 
	\RequirePackage{notoccite}
% \end{macrocode}
% 
% 
% \AmSTeX\ 宏包,用于排版出更加漂亮的公式。
% \begin{macrocode}
	\RequirePackage{amsmath}
% \end{macrocode}
% 
% 使用 \pkg{unicode-math} 处理数学字体。
% \begin{macrocode} 
	\RequirePackage{unicode-math}
% \end{macrocode}
% 
% 使用 \pkg{graphicx} 提供图形支持
% \begin{macrocode} 
	\RequirePackage{graphicx}
% \end{macrocode}
% 
% 使用 \pkg{subcaption} 
% \begin{macrocode} 
	\RequirePackage[labelformat=simple]{subcaption}
% \end{macrocode}
% 
% 使用 \pkg{pdfpages} 便于插入扫描后的授权说明和声明页 PDF 文档。
% \begin{macrocode} 
	\RequirePackage{pdfpages}
	\includepdfset{fitpaper=true}
% \end{macrocode}
% 
% 
% 列表环境
% \begin{macrocode} 
	\RequirePackage[shortlabels]{enumitem}
	\RequirePackage{environ} 
% \end{macrocode}
% 
% 禁止 \LaTeX{} 自动调整多余的页面底部空白,,并保持脚注仍然在底部。
% 脚注按页编号。。
% \begin{macrocode} 
	\ifnuc@raggedbottom
	\RequirePackage[bottom,perpage,hang]{footmisc}
	\raggedbottom 
	\else
	\RequirePackage[perpage,hang]{footmisc}
	\fi
% \end{macrocode}
% 
% 使用 \pkg{xeCJKfntef} 实现汉字的下划线和盒子内两端对齐,,并可以避免
% \cs{makebox}\oarg{width}\oarg{s} 可能产生的 underful boxes.
% \begin{macrocode} 
	\RequirePackage{xeCJKfntef}
	\RequirePackage{soul}
% \end{macrocode}
% 
% 表格控制 
% \begin{macrocode} 
	\RequirePackage{array}
% \end{macrocode}
% 
% 使用三线表: \cs{toprule},\cs{midrule} \cs{bottomrule}
% \begin{macrocode} 
	\RequirePackage{booktabs}
% \end{macrocode}
% 
% 
% \begin{macrocode} 
	\RequirePackage{url} 
% \end{macrocode}
% 
% 如果用户未在导言区调用 \pkg{biblatex},,则默认使用 \pkg{natbib}
% \begin{macrocode} 
	\AtEndPreamble{
		\@ifpackageloaded{biblatex}{}{
			\@ifpackageloaded{apacite}{}{
				\RequirePackage{natbib}
			}
		}
	}
	\AtEndOfPackageFile*{natbib}{
		\@ifpackageloaded{apacite}{}{
			\RequirePackage{bibunits}
		}
	}

% \end{macrocode}
% 
% 对冲突的宏包报错。
% \begin{macrocode} 
	\newcommand\nuc@package@conflict[2]{
		\AtBeginOfPackageFile*{#2}{
			\nuc@error{The "#2" package is incompatible with required "#1"}
		}
	}
	\nuc@package@conflict{unicode-math}{amscd}
	\nuc@package@conflict{unicode-math}{amsfonts}
	\nuc@package@conflict{unicode-math}{amssymb}
	\nuc@package@conflict{unicode-math}{bbm}
	\nuc@package@conflict{unicode-math}{bm}
	\nuc@package@conflict{unicode-math}{eucal}
	\nuc@package@conflict{unicode-math}{eufrak}
	\nuc@package@conflict{unicode-math}{mathrsfs}

% 禁止导入 setspace 包 
\AtBeginOfPackageFile*{setspace}{
	\nuc@error{The "setspace" package is incompatible with table environment, which will make invalidate the float format}
}
% \end{macrocode}
% 
% 
% \subsection{页面设置}
% \label{sec:layout}
% 研究生 《写作指南》::
% 页边距:上边距为 3.5 厘米,下边距为 3 厘米,左边距为 3 厘米,下边为 2 厘米。
% 页眉距边界:2.5 厘米,页脚距边界:1.8 厘米。
% 
% 本科生 《写作指南》
% % 页边距:上边距为 2.5 厘米,其余均为 2 厘米。
% 页眉距边界:2.0 厘米,页脚距边界:1.5 厘米。
%
% 
%  \pkg{fancyhdr} 的页眉是沿底部对齐的,所以只需设置 \cs{headsep},
% \cs{headheight} 可以适当增加高度允许多行页眉。
% 
% 研究生:\cs{headsep} = $\SI{3.5}{cm} - $\SI{2.5}{cm} - \SI{10.5}{bp} \times 1.3
% \approx \SI{0.3}{cm}
% \begin{macrocode} 
	\geometry{
		paper = a4paper, % 210 * 297mm
		marginparwidth = 2cm, % 侧边注释区域宽度
		marginparsep = 0.5cm, % 侧边注释与正文之间的宽度
	}
	\newcommand\nuc@set@geometry{
		\ifnuc@degree@bachelor{
			\geometry{
				top        = 2.5cm,
				bottom     = 2.0cm,
				left       = 2.0cm,
				right      = 2.0cm,
				headheight = 1.9cm,
				headsep    = 1.9cm,
				footskip   = 1.45cm, 
			}
			\ifnuc@output@print
				\geometry{
					left   = 3.2cm,
					right  = 3cm,
				}
			\else 
				\geometry{
					hmargin= 3cm,
				}
			\fi 
		\else 
			\geometry{
				top        = 3.5cm,
				bottom     = 3.0cm,
				left       = 3.0cm,
				right      = 2.0cm,
				headheight = 2.7cm,
				headsep    = 0.5cm,
				footskip   = 

			}
		}
	}
 








%  
