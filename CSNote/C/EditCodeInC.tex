\section{Planning}
大部分的新手,在编写程序时通过不做规划,导致在最后的花费大量的时间才重新修正程序。所以编写程序的首要任务是做好规划,这样做可以减少之后的程序调试 (Debug) 时间。

一个良好算法程序的诞生需要经历以下几个阶段
\begin{enumerate}
	\item (Work an Example yourself) 自己做一个例子。
	\item (Write down what you just did) 记录你刚才的工作。
	\item (Generalize you steps) 总结前两个步骤中的工作。
	\item (Test you steps ) 测试结论。
	\item (Translate to code) 将结论转为代码形式。
	\item (Test you Program) 测试程序是否符合结论。
	\item (Debug your Program) 调试代码。

\end{enumerate}

\section{Compiliing (编译)}
编译程序就是将人类编写的代码变为计算机可执行的形式。

.c 文件编译后会生成.out文件,

编译命令
gcc -c xyz.c 会将xyz.c 文件编译为 xyz.o 文件。(如果你希望编译后的文件能换一个名字,需要加上 -0)


\begin{figure}
	\centering
	\includegraphic{./figures/compileCFile.jpg}
	\label{fig:compileCFile}
	
\end{figure}



\subsection{C文件各部分介绍}
\subsubsection{头文件 (Header Files)}
头文件是指包含 \textbf{#include}的部分。
其中 \textbf{#include} 之后加 \textbf{<>} 是指:引入的文件是C语言的标准头文件 (头文件中通常包含三项内容:函数原型(function prototypes),宏定义(macro definitions)和类型声明(type declarations))。
若是之后加 \textbf{""} 是指:引入的文件是自己编写的头文件 (非标准头文件)


\cs{宏定义 Macros} 







\section{汇编 Assembling}
汇编就是通过编译器 (gcc) 将人类可理解的程序转换称为处理器可以理解和执行的数字编码。


\section{链接 Linking}
链接程序需要将一个或多个文件结合在一起,生成实际的可执行的二进制文件。

对于C语言自己的库,通常情况下是默认链接的。如果你希望使用其他非C的库,你必须使用 \textbf{-l} 实现对目标文件的链接。

\section{make}
平时我们编译少量的文件我们常用的操作为使用gcc编译器对代码文件进行编译。但对于开发一个大型应用如果继续采用 gcc编译器,那么效果就不尽人意了。此时,make工具出现了。make工具可以自动的解决文件修改后的链接问题。

make通过默认读取\textbf{Makefile}文件,在这个文件中指定了如何编译文件。

\section{Testing and debugging}
Testing 是为了找到bugs ,而 debugging 是为了修改bugs。

按照测试的不同特点,我们将测试分为以下几类:
\begin{enumerate}
	\item 黑盒测试
	\item 白盒测试
			
\end{enumerate}

\subsection{黑盒测试} 

黑盒测试也叫功能测试,通过测试确定每一个功能是否都能正常使用。在测试过程中,将程序看作一个不能打开的黑盒子,用户是看不到程序的内部结构的,只能对程序接口进行测试,对比开发文档和输出结果判断程序是否正常。

黑盒测试过程中,测试人员要尽可能的考虑哪些情况会导致问题的出现。

设计测试案例的技巧
\begin{enumerate}
	\item 测试错误案例的一些想法
		\begin{itemize}
			\item 确保测试覆盖所有的错误情况。
			\item 测试越界后的案例。
			\item 在有效边界进行测试。
		\end{itemize}

	\item 测试输入的建议
		\begin{itemize}
			\item 考虑是否有特殊的情况
			\item 考虑是否有令人误解的情况
			\item 考虑所有的类型,如果开发人员使用的错误的类型,则会发生错误情况。
			\item 对于数值输入,考虑正数、负数和零
			\item 对于序列输入,考虑空序列、单元素序列和多元素序列
			\item 对于字符输入,考虑小写字符、大写字符、数字、标点符号、空格和不可打印字符。
			\item 
		\end{itemize}
	

\end{enumerate}


\subsection{白盒测试}


