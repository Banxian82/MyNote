\section{分析化学}

\begin{itemize}
	\item 什么是分析化学: 测定物质的组成、含量、结构的一门学科。
	\item 它的任务是什么

	\begin{itemize}
		\item 测定组成:定性分析
		\item 测定含量:定量分析
		\item 测定结构:结构分析	
	\end{itemize}

	\item 作用: 
	
\end{itemize}

分析方法的分类及选择原则

分析方法分类
\begin{itemize}
	\item 依据分析工作的任务不同分为
	\begin{itemize}
		\item 定性分析
		\item 定量分析
		\item 结构分析
	\end{itemize}
		
	\item 分为
	\begin{itemize}
		\item 化学分析法
		\begin{itemize}
			\item 容量分析方法 (滴定分析法 (包括酸碱滴定法、氧化还原滴定法、沉淀滴定法、络合滴定法(也称配位滴定法))) 
			\item 重量分析方法 (缺点:麻烦)
				
		\end{itemize}
		\item 仪器分析法 (依据物质的性质和某一个组分建立的关系) (测定物质含量低的测定方法)

	\end{itemize}
\end{itemize}
\section{分析过程及分析结果表示}
\subsection{分析过程}
取样 -> 预处理() -> 分析测试 -> 得到分析数据 -> 数据处理 -> 确定质量(从精密度和准确度看出质量好坏) -> 报告

\subsection{分析结果表示}
\begin{enumerate}
	\item 化学形式
	\begin{enumerate}
		\item 元素符号。例如:N、P.
		\item 实际存在形式 $cl^{-1}$
		\item 氧化物形式 例如 $Fe_2O_3$ 
		
	\end{enumerate}

	\item 含量形式
	\begin{enumerate}
		\item 固体试样 质量分数
		\item 液体试样 物质的量浓度(摩尔浓度 mol/L) 、质量摩尔浓度(mol/g)、质量浓度 (mg/L)、质量分数、体积分数
		\item 气体试样 质量浓度 (mg/L)、体积分数 ()
		
	\end{enumerate}
	
\end{enumerate}

\subsection{滴定实验}

滴定分析法术语
\begin{itemize}
	\item 滴定反应
	\item 滴定指示剂
	\item 标准溶液
	\item 滴定
	\item 化学计量点 
	\item 滴定剂
	\item 滴定终点 (滴定达到指示剂 (酚酞) 变色)
	\item 滴定误差 ( |TE| <= 0.1\% )
	
\end{itemize}

滴定分析法:将标准溶液滴加到被测物质溶液中,二者发生滴定反应,按照指示剂的指示并根据标准溶液的消耗量来计算被测物质的量。

滴定反应要求
\begin{enumerate}
	\item 速度快
	\item 反应定量完成
	\item 有合适的指示终点方法
	
\end{enumerate}

滴定方式 

假设滴定剂为A,被测溶液为B,其他溶液为C。
\begin{enumerate}
	\item 直接滴定方式 : (A 滴 B) 滴定剂和被测物质直接反应
	\item 间接滴定方式 : (A 滴 (B+C) ) 被测溶液中加入另外一种物质,使其反应生成物质A,然后再使用滴定剂与物质A进行反应。 
	例如: 滴定剂为 $KM_2O_4$,被测溶液为:含$Ca^{2+}$,我们需要测定被测溶液中$Ca^{2+}$的含量,我们加入草酸 ($H_2C_2O_4$),使得草酸与钙离子发生反应得到草酸钙,然后使用高锰酸钾滴定被测溶液中的草酸根离子 ($C_2O_4^{2-}$) 反应,从而间接得知钙离子的含量。
	\begin{equation}
		ca^{2+} + KM_2O_4 
	\end{equation}

	\item 返滴定方式: (C 滴 (A + B))
	\item 置换滴定方式 : (A 滴 (B + C))
	
\end{enumerate}

\subsection{标准溶液和基准物质}

\subsubsection{基准物质}
基准物质:可以用于直接配制标准溶液的物质。

常用的基准物质:
\begin{itemize}
	\item 
	
\end{itemize}

基准物质要求
\begin{enumerate}
	\item 
	\item 稳定
	\item 式量大
	
\end{enumerate}


\subsubsection{标准溶液}

\begin{enumerate}
	\item 直接法配置。基准物质 -> 标准溶液 (计算出量后称量)
	\item 标定法 (也叫间接配置法)。 
	\begin{enumerate}
		\item 粗配

		
	\end{enumerate}
	
\end{enumerate}

\subsection{滴定分析计算}

\subsubsection{标准溶液表示方法}


\begin{enumerate}
	\item 基本单元。 酸碱反应以能够提供一个氢离子或结合一个氢氧根离子为一个基本单元,氧化还原反应以提供一个电子或得到一个电子为一个基本单元。 
	
\end{enumerate}


