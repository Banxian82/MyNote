\iffalse meta-comment

Copyright (C) 2025 by Norbury <NorburyMJ@outlook.com>


\fi


\iffalse
%<*driver>
\ProvidesFile{MyNoteThesis.dtx}[2025 Note Thesis Template]
\documentclass{ltxdoc}
\usepackage{dtx-style}
\EnableCrossrefs
\CodelineIndex
\begin{document}
    \DocInput{\jobname.dtx}
\end{document}
%</driver>
% 

\fi 

% \DoNotIndex{\newenvironment,\@bsphack,\@empty,\@esphack,\sfcode}
% \DoNotIndex{\addtocounter,\label,\let,\linewidth,\newcounter}
% \DoNotIndex{\noindent,\normalfont,\par,\parskip,\phantomsection}
% \DoNotIndex{\providecommand,\ProvidesPackage,\refstepcounter}
% \DoNotIndex{\RequirePackage,\setcounter,\setlength,\string,\strut}
% \DoNotIndex{\textbackslash,\texttt,\ttfamily,\usepackage}
% \DoNotIndex{\begin,\end,\begingroup,\endgroup,\par,\\}
% \DoNotIndex{\if,\ifx,\ifdim,\ifnum,\ifcase,\else,\or,\fi}
% \DoNotIndex{\let,\def,\xdef,\edef,\newcommand,\renewcommand}
% \DoNotIndex{\expandafter,\csname,\endcsname,\relax,\protect}
% \DoNotIndex{\Huge,\huge,\LARGE,\Large,\large,\normalsize}
% \DoNotIndex{\small,\footnotesize,\scriptsize,\tiny}
% \DoNotIndex{\normalfont,\bfseries,\slshape,\sffamily,\interlinepenalty}
% \DoNotIndex{\textbf,\textit,\textsf,\textsc}
% \DoNotIndex{\hfil,\par,\hskip,\vskip,\vspace,\quad}
% \DoNotIndex{\centering,\raggedright,\ref}
% \DoNotIndex{\c@secnumdepth,\@startsection,\@setfontsize}
% \DoNotIndex{\ ,\@plus,\@minus,\p@,\z@,\@m,\@M,\@ne,\m@ne}
% \DoNotIndex{\@@par,\DeclareOperation,\RequirePackage,\LoadClass}
% \DoNotIndex{\AtBeginDocument,\AtEndDocument}
% 

\GetFileInfo{\jobname.dtx}

\def\indexname{索引}
\IndexPrologue{\section{\indexname}}

\title{\bfseries\color{violet}
    \notethesis : 笔记模板
}
\author{{\fangsong Norbury}[5pt]\texttt{NorburyMJ@outlook.com}}
\date{v\fileversion\ (\filedata)}
\maketitle\thispagestyle{empty}


\section{实现细节}

\subsection{基本信息}
\begin{macrocode}
    <cls>\NeedsTeXFormat{LaTeX2e}[2019]
    <cls>\ProvidesClass{notethesis}
    <cls>[2025/07/01 note thesis template]
\end{macrocode}

报错
\begin{macrocode}
    \newcommand\note@error[1]{
        \ClassError{notethesis}{#1}{}
    }
    \newcommand\note@warning[1]{
        \ClassWarning{notethesis}[#1]
    }
    \newcommand\note@patch@error[1]{
        \note@error{Failed to patch command \protect#1}
    }
    \newcommand\note@deprecate[2]{
        \def\note@@tmp{#2}
        \note@warning{
            The #1 is deprecated
            \ifx\note@@tmp\@empty\else
                .Use #2 instead
            \fi
        }
    }
\end{macrocode}


检查 \LaTeXe{} kernel 版本

\begin{macrocode}
    \@ifl@t@r\fmtversion{2020/01/01}{}{
        \note@error{
            TeX Live 2020 or later version is required to compile this document.
        }
    }
\end{macrocode}

检查编译引擎,要求使用 \XeLaTeX 
\begin{macrocode}
    \RequirePackage{iftex}
    \ifXeTeX\else
        \note@error{XeLaTeX is required to compile this document}
    \fi
\end{macrocode}

\subsection{定义选项}
\label{sec:defoption}
定义论文类型以及是否涉密
\begin{macrocode}
    <*cls>
    \hyphenation{Note-Thesis} % 告诉 TeX 在断词时如何正确地在连字符处断开这些单词。
    \def\notethesis{NoteThesis}
    \def\version{0.0.1}
    \RequirePackage{kvdefinekeys}
    \RequirePackage{kvsetkeys}
    \RequirePackage{kvoptions}
    \SetupKeyValOptions{
        family=note,
        prefix=note@,
        setkeys=\kvsetkeys
    }
     
\end{macrocode}

\begin{macro}{\notesetup}
    提供一个 \cs{notesetup} 命令支持 \emph{key-value} 的方式设置。
    \begin{macrocode}
        \newcommand\notesetup[1]{
            \kvsetkeys{note}{#1}
        }
    \end{macrocode}

\end{macro}

% 同时用 \emph{key-value} 的方式来定义这些接口:
% \begin{latex}
%   \note@define@key{
%     <key> = {
%       name = <name>,
%       choices = {
%         <choice1>,
%         <choice2>,
%       },
%       default = <default>,
%     },
%   }
% \end{latex}
% 

\begin{macrocode}
    \newcommand\note@define@key[1]{
        \kvsetkeys{note@key}{#1}
    }
    \kv@set@family@handler{note@key}{}
\end{macrocode}