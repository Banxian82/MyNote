\section{溶液 aqueous solution}
溶液是一种均匀混合物,通常由溶质和溶剂组成。

通常我们认为的溶液仅仅是液体,但是固体和气体也能称为溶液,例如标准纯银是固溶体,空气是气溶体。

在本章我们要探究,一种物质溶解到另一种物质之中会发生什么、为什么会形成溶液,有什么因素可以影响到溶液的形成以及形成速率,溶液有什么性质,这些性质有什么作用、能够应用到哪些方面。

\subsection{溶解过程}
物质形成溶液的能力取决于两种因素
\begin{itemize}
	\item 物质在不受约束的情况下混合和扩散到更大体积的趋势
	\item 溶解过程中分子间相互作用力的类型
	
\end{itemize}

我们将溶液的类型分为固态溶液、气态溶液以及液态溶液。
其中气态溶液会自发的相互混合形成气态溶液 (即混合过程不需要外界能量的提供即可进行),而当溶质或溶剂为固体或者液体时,它们无法自发的进行混合。这时分子间作用力就显得尤为重要,当溶质-溶剂之间的分子间作用力能够克服溶剂-溶剂和溶质-溶质之间的分子间作用力,那么它们更容易形成溶液。

溶液中的分子间作用力
\begin{itemize}
	\item 离子-偶极子相互作用力,
	
\end{itemize}



