\section{课后题}

\subsection{思考题}

\begin{enumerate}
	\item 简述分析化学的定义、任务和作用

		分析化学的定义: 用于测定物质的组成、含量和结构。
		分析化学的任务:

	\item 讨论选择分析方法的原则

		\begin{itemize}
			\item 根据所分析物质的组成、含量确定分析方法
			\item 获取共存组分的信息并考虑共存组分对测定的影响,拟定合适的分离富集方法,以提高分析方法的选择性
			\item 对测定准确度、灵敏度的要求和对策。
			\item 现有条件、测定成本及完成测定的时间要求等
			
		\end{itemize}

	\item 简述一般试样的分析过程

		取样 \rightarrow 预处理 \rightarrow 分析测试 \rightarrow 得到分析数据 \rightarrow 数据分析 \rightarrow 确定质量 \rightarrow 报告

	\item 

		邻苯二甲酸氢钾(KHC_8H_4O_4) 更为合适,因为该物质中没有其他共存组分。而二水合草酸(H_2C_2O_4 \dot 2H_2O) 中有结晶水的存在,可能会造成影响。


	\item 
		选择 Na_2CO_3 更加适合。排除了结晶水的存在会造成的影响。

	\item
		a. 偏低
		b. 偏高
		c. 偏低
		d. 偏高
		e. 不知
		f. 无影响
		g. 无影响
		h. 偏低

	\item 
		无影响

	\item 
		无影响

\end{enumerate}

\subsection{习题}

\begin{question}
	zn + 2$cl^{-1}$ = $zn^{2+}$

	1. 计算 zn 的物质的量(n 单位 mol)
	n_{zn} & = m_{zn}/M_{zn} 
	& = 0.3250 g / 65.38 g \dot mol^{-1}  
	& = 0.0050 mol 

	2. 分析 $zn^{2+}$ 与 zn 的物质的量的关系
	$n(zn^{2+} = n(zn) = 0.0050 mol) $

	3. 计算溶液体积
	V_{ZnCl} = 250 mL = 0.250 L 

	4. 计算 $Zn^{2+}$ 溶液的浓度
	$c_{Zn^{2+}} & = \frac{n_{Zn^{2+}} mol}{V_{Zn^{2+}} L} $
	& = \frac{0.0050 mol}{0.250 L}
	& = 0.02 mol/L
	
\end{question}

\begin{question}
	改变溶液浓度时,溶质的物质的量 (或质量) 在仅稀释 / 浓缩且溶质无增减的情况下保持不变。

	1. 计算 原来 $H_2SO_4$ 中的物质的量
	$n_{H_2SO_4} & = 0.0982 mol \dot L^{-1} \times 480 mL $ 
	& = 0.0470 mol 

	2. 根据浓缩后 物质的量(n) 不变,设立如下方程

	0.0982 mol \dot L^{-1} \times 0.480 L + 0.5000 mol \dot L^{-1} \times V_{H_2SO_4} = 0.1000 mol \dot L^{-1} \times (480 mL + V_{H_2SO_4})

	3. 求解体积 $V_{H_2SO_4}$
	V_{H_2SO_4} = \frac{0.1000 mol \dot L^{-1} \times 480 mL - 0.0982 mol \dot L^{-1} \times 480 mL}{0.5000 mol \dot L^{-1} - 0.1000 mol \dot L^{-1}}



	
\end{question}

\begin{question}
	1. 计算 $K_4Fe(CN)_6$ 的摩尔质量 (数值上相当于相对分子质量) 
	$n_{K_4Fe(CN)_6}$ = 368 g/mol 

	2. 计算该溶液的浓度 \\ 
	$c_{K_4Fe(CN)_6} &= m_{K_4Fe(CN)_6} / n_{K_4Fe(CN)_6} V$ \\ 
	& = \frac{9.21 g}{368 g/mol \times 0.500 L}\\ 
	& = 0.0500 mol / L 

	3. 根据反应方程式确定标准溶液与被测离子的反应方程式,明确二者之间的物质的量之比 \\ 
	$n_{Zn^{2+}} / n_{K_4Fe(CN)_6} = 2/3 $

	4. 计算对 $Zn^{2+}$ 的滴定度(T) \\ 
	T & = $ c \times V_{标液} \times \frac{n_{被测离子}}{n_{标液}}  \times M_{被测离子}$
	& = 0.05 mol/L \times 0.001 L \times 1.5 \times 65.38 
	& = 4.9 mg/mL 
\end{question}

\begin{question}
	1. 计算邻苯二甲酸氢钠的摩尔质量 \\
	$n_{KHC_8H_4O_4}$ = 204 g/mol 

	2. 计算称取基准试剂邻苯二甲酸氢钾的克重\\
	$m_1 & = n_{KHC_8H_4O_4} \times c_{NaOH} \times V_{NaOH} $ \\ 
	& = 204 g/mol \times 0.2 mol/L \times 0.025 L \\
	& = 1.0g 

	$m_2 & = 204 g/mol \times 0.2 mol/L \times 0.03 L $ \\
	& = 1.2g

	3. 改用 $H_2C_2O_4 \dot 2H_2O $ 作为基准物质后,计算 $H_2C_2O_4 \dot 2H_2O$ 的摩尔质量\\ 
	$n_{H_2C_2O_4 \dot 2H_2O} = 126 g/mol $\\ 

	4. 确定物质的量的比例,并计算称取二水合草酸的克重\\ 
	氢氧化钠和二水合草酸反应的比例为2:1 \\ 
	$m_{H_2C_2O_4 \dot 2H_2O}  & = \frac{n_{H_2C_2O_4 \dot 2H_2O} \times c_{NaOH} \times V_{NaOH}}{2} $ \\ 
	& = \frac{130 g/mol \times 0.2 mol/L \times 0.025 L}{2} \\
	& =  0.3 g 

	上限为 & = \frac{126 g/mol \times 0.2 mol/L \times 0.03 L}{2} \\
	& = 0.4 g

	
\end{question}


\begin{question}

	
\end{question}
