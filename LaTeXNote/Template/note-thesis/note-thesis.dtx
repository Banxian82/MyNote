\iffalse meta-comment

Copyright (C) 2025 by Norbury <NorburyMJ@outlook.com>


\fi


\iffalse
%<*driver>
\ProvidesFile{MyNoteThesis.dtx}[2025 Note Thesis Template]
\documentclass{ltxdoc}
\usepackage{dtx-style}
\EnableCrossrefs
\CodelineIndex
\begin{document}
    \DocInput{\jobname.dtx}
\end{document}
%</driver>
% 

\fi 

% \DoNotIndex{\newenvironment,\@bsphack,\@empty,\@esphack,\sfcode}
% \DoNotIndex{\addtocounter,\label,\let,\linewidth,\newcounter}
% \DoNotIndex{\noindent,\normalfont,\par,\parskip,\phantomsection}
% \DoNotIndex{\providecommand,\ProvidesPackage,\refstepcounter}
% \DoNotIndex{\RequirePackage,\setcounter,\setlength,\string,\strut}
% \DoNotIndex{\textbackslash,\texttt,\ttfamily,\usepackage}
% \DoNotIndex{\begin,\end,\begingroup,\endgroup,\par,\\}
% \DoNotIndex{\if,\ifx,\ifdim,\ifnum,\ifcase,\else,\or,\fi}
% \DoNotIndex{\let,\def,\xdef,\edef,\newcommand,\renewcommand}
% \DoNotIndex{\expandafter,\csname,\endcsname,\relax,\protect}
% \DoNotIndex{\Huge,\huge,\LARGE,\Large,\large,\normalsize}
% \DoNotIndex{\small,\footnotesize,\scriptsize,\tiny}
% \DoNotIndex{\normalfont,\bfseries,\slshape,\sffamily,\interlinepenalty}
% \DoNotIndex{\textbf,\textit,\textsf,\textsc}
% \DoNotIndex{\hfil,\par,\hskip,\vskip,\vspace,\quad}
% \DoNotIndex{\centering,\raggedright,\ref}
% \DoNotIndex{\c@secnumdepth,\@startsection,\@setfontsize}
% \DoNotIndex{\ ,\@plus,\@minus,\p@,\z@,\@m,\@M,\@ne,\m@ne}
% \DoNotIndex{\@@par,\DeclareOperation,\RequirePackage,\LoadClass}
% \DoNotIndex{\AtBeginDocument,\AtEndDocument}
% 

% \GetFileInfo{\jobname.dtx}

% \def\indexname{索引}
% \IndexPrologue{\section{\indexname}}

% \title{\bfseries\color{violet}
%     \notethesis : 笔记模板
% }
% \author{{\fangsong Norbury}[5pt]\texttt{NorburyMJ@outlook.com}}
% \date{v\fileversion\ (\filedata)}
% \maketitle\thispagestyle{empty}


% \section{实现细节}

% \subsection{基本信息}
% \begin{macrocode}
%     <cls>\NeedsTeXFormat{LaTeX2e}[2019]
%     <cls>\ProvidesClass{notethesis}
%     <cls>[2025/07/01 note thesis template]
% \end{macrocode}

% 报错
% \begin{macrocode}
    \newcommand\note@error[1]{
        \ClassError{notethesis}{#1}{}
    }
    \newcommand\note@warning[1]{
        \ClassWarning{notethesis}[#1]
    }
    \newcommand\note@patch@error[1]{
        \note@error{Failed to patch command \protect#1}
    }
    \newcommand\note@deprecate[2]{
        \def\note@@tmp{#2}
        \note@warning{
            The #1 is deprecated
            \ifx\note@@tmp\@empty\else
                .Use #2 instead
            \fi
        }
    }
% \end{macrocode}


% 检查 \LaTeXe{} kernel 版本

% \begin{macrocode}
    \@ifl@t@r\fmtversion{2020/01/01}{}{
        \note@error{
            TeX Live 2020 or later version is required to compile this document.
        }
    }
% \end{macrocode}

% 检查编译引擎,要求使用 \XeLaTeX 
% \begin{macrocode}
    \RequirePackage{iftex}
    \ifXeTeX\else
        \note@error{XeLaTeX is required to compile this document}
    \fi
% \end{macrocode}

% \subsection{定义选项}
% \label{sec:defoption}
% 定义论文类型以及是否涉密
% \begin{macrocode}
    <*cls>
    \hyphenation{Note-Thesis} % 告诉 TeX 在断词时如何正确地在连字符处断开这些单词。
    \def\notethesis{NoteThesis}
    \def\version{0.0.1}
    \RequirePackage{kvdefinekeys}
    \RequirePackage{kvsetkeys}
    \RequirePackage{kvoptions}
    \SetupKeyValOptions{
        family=note,
        prefix=note@,
        setkeys=\kvsetkeys
    }
     
% \end{macrocode}

% \begin{macro}{\notesetup}
%     提供一个 \cs{notesetup} 命令支持 \emph{key-value} 的方式设置。
%     \begin{macrocode}
        \newcommand\notesetup[1]{
            \kvsetkeys{note}{#1}
        }
%     \end{macrocode}

% \end{macro}

% 同时用 \emph{key-value} 的方式来定义这些接口:
% \begin{latex}
%   \note@define@key{
%     <key> = {
%       name = <name>,
%       choices = {
%         <choice1>,
%         <choice2>,
%       },
%       default = <default>,
%     },
%   }
% \end{latex}
% 

% \begin{macrocode}
    \newcommand\note@define@key[1]{
        \kvsetkeys{note@key}{#1}
    }
    \kv@set@family@handler{note@key}{}
% \end{macrocode}

% \begin{macrocode}
    \@namedef{note@#1@@name}{#1}
    \def\note@@default{}
    \def\note@@choices{}
    \kv@define@key{note@value}{name}{
        \@namedef{note@#1@@name}{##1}
    }
% \end{macrocode}


% 定义键值选项时,为每一个选项创建对应的“检查”和“执行”宏
% \begin{macrocode}
    \@namedef{note@#1@@check}{}
    \@namedef{note@#1@@code}{}
% \end{macrocode}

% 保存下 |choices = {}| 定义的内容,在定义 \cs{note@\meta{name}} 后再执行。
%
%    \begin{macrocode}
  \kv@define@key{note@value}{choices}{%
    \def\note@@choices{##1}%
    \@namedef{note@#1@@reset}{}%
%    \end{macrocode}
%
% \cs{note@\meta{key}@check} 检查 |value| 是否有效,
% 并设置 \cs{ifnote@\meta{name}@\meta{value}}。
%
%    \begin{macrocode}
    \@namedef{note@#1@@check}{%
      \@ifundefined{%
        ifnote@\@nameuse{note@#1@@name}@\@nameuse{note@\@nameuse{note@#1@@name}}%
      }{%
        \note@error{Invalid value "#1 = \@nameuse{note@\@nameuse{note@#1@@name}}"}%
      }%
      \@nameuse{note@#1@@reset}%
      \@nameuse{note@\@nameuse{note@#1@@name}@\@nameuse{note@\@nameuse{note@#1@@name}}true}%
    }%
  }%
  \kv@define@key{note@value}{default}{%
    \def\note@@default{##1}%
  }%
  \kvsetkeys{note@value}{#2}%
  \@namedef{note@\@nameuse{note@#1@@name}}{}%
%    \end{macrocode}
%
% 第一个 \meta{choice} 设为 \meta{default},
% 并且对每个 \meta{choice} 定义 \cs{ifnote@\meta{name}@\meta{choice}}。
%
%    \begin{macrocode}
  \kv@set@family@handler{note@choice}{%
    \ifx\note@@default\@empty
      \def\note@@default{##1}%
    \fi
    \expandafter\newif\csname ifnote@\@nameuse{note@#1@@name}@##1\endcsname
    \expandafter\g@addto@macro\csname note@#1@@reset\endcsname{%
      \@nameuse{note@\@nameuse{note@#1@@name}@##1false}%
    }%
  }%
  \kvsetkeys@expandafter{note@choice}{\note@@choices}%
%    \end{macrocode}
%
% 将 \meta{default} 赋值到 \cs{note@\meta{name}},如果非空则执行相应的代码。
%
%    \begin{macrocode}
  \expandafter\let\csname note@\@nameuse{note@#1@@name}\endcsname\note@@default
  \expandafter\ifx\csname note@\@nameuse{note@#1@@name}\endcsname\@empty\else
    \@nameuse{note@#1@@check}%
  \fi
%    \end{macrocode}
%
% 定义 \cs{notesetup} 接口。
%
%    \begin{macrocode}
  \kv@define@key{note}{#1}{%
    \@namedef{note@\@nameuse{note@#1@@name}}{##1}%
    \@nameuse{note@#1@@check}%
    \@nameuse{note@#1@@code}%
  }%

%    \end{macrocode}
%
% 定义接口向 |key| 添加 |code|:
%
%    \begin{macrocode}
\newcommand\note@option@hook[2]{%
  \expandafter\g@addto@macro\csname note@#1@@code\endcsname{#2}%
}
%    \end{macrocode}

% 设置学位类型
\begin{macrocode}
	degree = {
		choices = {
			bachelor,
			master,
			doctor,
			postdoc,
		},
		
		degree-type = {
			default = academic,
			choices = {
				academic,
				professional,
			},
		name = degree@type,
	},
	
\end{macrocode}


论文的主要语言
\begin{macrocode}
	main-language = {
		name = main@language,
		choices = {
			chinese,
			english,
		},
	},
	
\end{macrocode}

字体
%\begin{macrocode}
  fontset = {
    choices = {
      windows,
      mac,
      ubuntu,
      fandol,
      none,
    },
    default = none,
  },
  system = {
    choices = {
      mac,
      unix,
      windows,
      auto,
    },
    default = auto,
  },
  font = {
    choices = {
      times,
      termes,
      auto,
      none,
      external,
    },
    default = auto,
  },
  cjk-font = {
    name = cjk@font,
    choices = {
      windows,
      mac,
      noto,
      fandol,
      auto,
      none,
      external,
    },
    default = auto,
  },
  math-font = {
    name = math@font,
    choices = {
      cambria,
      xits,
      stix,
      none,
      times,
    },
    default = cambria,
  },
  math-style = {
    name = math@style,
    choices = {
      GB,
      TeX,
    },
  },
%\end{macrocode}
%
% 选择打印版还是用于上传的电子版。
%\begin{macrocode}
  output = {
    choices = {
      print,
      electronic,
    },
    default = print,
  },
}
\newif\ifthu@degree@graduate
\newcommand\note@set@graduate{%
  \note@degree@graduatefalse
  \ifthu@degree@doctor
    \note@degree@graduatetrue
  \fi
  \ifthu@degree@master
    \note@degree@graduatetrue
  \fi
}
\note@set@graduate
\note@option@hook{degree}{%
  \note@set@graduate
}
%    \end{macrocode}
%
% 设置默认 \option{openany}。
%    \begin{macrocode}
\DeclareBoolOption[false]{openright}
\DeclareComplementaryOption{openany}{openright}
%    \end{macrocode}
%
% \option{raggedbottom} 选项(默认打开)
%    \begin{macrocode}
\DeclareBoolOption[true]{raggedbottom}
%    \end{macrocode}
%
% 将选项传递给 \pkg{ctexbook}。
%    \begin{macrocode}
\DeclareDefaultOption{\PassOptionsToClass{\CurrentOption}{ctexbook}}
%    \end{macrocode}
%
% 解析用户传递过来的选项,并加载 \pkg{ctexbook}。
%    \begin{macrocode}
\ProcessKeyvalOptions*
%    \end{macrocode}
%
% 设置默认 \option{openany}。
%    \begin{macrocode}
\ifthu@openright
  \PassOptionsToClass{openright}{book}
\else
  \PassOptionsToClass{openany}{book}
\fi
%    \end{macrocode}
%
% 使用 \pkg{ctexbook} 类,优于调用 \pkg{ctex} 宏包。
%    \begin{macrocode}
\PassOptionsToPackage{quiet}{fontspec}
\LoadClass[a4paper,UTF8,zihao=-4,scheme=plain,fontset=none]{ctexbook}[2017/04/01]
%    \end{macrocode}
%
%
% \subsection{装载宏包}
% \label{sec:loadpackage}
%
% 引用的宏包和相应的定义。
%    \begin{macrocode}
\RequirePackage{etoolbox}
\RequirePackage{filehook}
\RequirePackage{xparse}
%    \end{macrocode}
%
% 利用 \pkg{textcase} 英文大小写转换。
%    \begin{macrocode}
\RequirePackage{textcase}
%    \end{macrocode}
%
%    \begin{macrocode}
\RequirePackage{geometry}%
%    \end{macrocode}
%
% 利用 \pkg{fancyhdr} 设置页眉页脚。
%    \begin{macrocode}
\RequirePackage{fancyhdr}
%    \end{macrocode}
%
%    \begin{macrocode}
\RequirePackage{titletoc}
%    \end{macrocode}

利用 \pkg{notoccite} 避免目录中引用编号混乱
\begin{macrocode}
	\RequirePackage{notoccite}	
\end{macrocode}

\AmSTeX\ 宏包,用来排版出漂亮的公式
\begin{macrocode}
	\RequirePackage{amamath}	
\end{macrocode}

使用 \pkg{unicode-math} 处理数学字体
\begin{macrocode}
	\RequirePackage{unicode-math}	
\end{macrocode}

图形支持宏包
\begin{macrocode}
	\RequirePackage{graphicx}	
\end{macrocode}

并排图形,使用\pkg{subcaption}
浮动图形和表格标题样式, \pkg{caption}
\begin{macrocode}
	\RequirePackage[labelformat=simple]{subcaption}	
\end{macrocode}

\pkg{pdfpages} 便于插入扫描后的授权说明书
\begin{macrocode}
	\RequirePackage{pdfpages}
	\includepdfset{fitpaper=true}
\end{macrocode}

更好的列表环境
\begin{macrocode}
	\RequirePackage[shortlabels]{enumitem}
	\RequirePackage{environ}
\end{macrocode}

禁止 \LaTeX{} 自动调整多余的页面底部空白,并保持脚注仍然在底部。脚注按页编号
\begin{macrocode}
	\ifnote@raggedbottom 
		\RequirePackage[bottom,perpage,hang]{footmisc}
		\raggedbottom
	\else 
		\RequirePackage[perpage,hang]{footmisc}
	\fi 
\end{macrocode}

利用 \pkg{xeCJKfntef} 实现汉字的下划线和盒子内两段对齐,并可以避免 \cs{makebox} \oarg{width}\oarg{s} 可能产生的 underful boxes
\begin{macrocode}
	\RequirePackage{xeCJKfntef}
	\RequirePackage{soul}
\end{macrocode}

表格控制
\begin{macrocode}
	\RequirePackage{array}	
\end{macrocode}

使用三线表 \cs{toprule} \cs{midrule} \cs{bottomrule}

\begin{macrocode}
	\RequirePackage{booktabs}	
\end{macrocode}

\begin{macrocode}
	\RequirePackage{url}	
\end{macrocode}

如果用户未在导言区调用 \pkg{biblatex} 则自动调用 \pkg{natbib}
\begin{macrocode}
	\AtEndPreamble{
		\@ifpackageloaded{biblatex}{}{
			\@ifpackageloaded{apacite}{}{
				\RequirePackage{natbib}
			}
		}
	}
	\AtEndOfPackageFile*{natbib}{
		\@ifpackageloaded{apacite}{}{
			\RequirePackage{bibunits}
		}
	}

\end{macrocode}

对冲突的宏包报错
\begin{macrocode}
	\newcommand\note@package@conflict[2]{
		\AtBeginOfPackageFile*{#2}{
			\note@error{The "#2" package is incompatible with required "#1"
		}
	}

	\note@package@conflict{unicode-math}{amscd}
	\note@package@conflict{unicode-math}{amsfonts}
	\note@package@conflict{unicode-math}{amssymb}
	\note@package@conflict{unicode-math}{bbm}
	\note@package@conflict{unicode-math}{eucal}
	\note@package@conflict{unicode-math}{eufrak}
	\note@package@conflict{unicode-math}{mathrsfs}

	\AtBeginOfPackageFile*{setspace}{
		\note@error{The "setspace" package is incompatible with table environment , which will make invalidate the float format}
\end{macrocode}

\subsection{页面设置}
\label{sec:layout}

% 研究生《写作指南》:
% 页边距:上下左右均为 3.0 厘米,装订线 0 厘米;
% 页眉距边界:2.2 厘米,页脚距边界:2.2 厘米。
%
% 本科生《写作指南》:
% 页边距:上:3.8 厘米,下:3.2 厘米,左右:3 厘米,装订线:左 0.2 厘米。
% 本科生 Word 模板:
% 无页眉,页脚距边界:1.75 厘米。
%
% \pkg{fancyhdr} 的页眉是沿底部对齐的,所以只需设置 \cs{headsep},
% \cs{headheight} 可以适当增加高度允许多行页眉。
% 研究生:\cs{headsep} = $\SI{3}{cm} - \SI{2.2}{cm} - \SI{10.5}{bp} \times 1.3
% \approx \SI{0.3}{cm}$。
%
%\begin{macrocode}
\geometry{
  paper          = a4paper,  % 210 * 297mm
  marginparwidth = 2cm,
  marginparsep   = 0.5cm,
}
\newcommand\note@set@geometry{%
  \ifthu@degree@bachelor
    \geometry{
      top        = 3.8cm,
      bottom     = 3.2cm,
      left       = 3.2cm,
      right      = 3cm,
      headheight = 1.9cm,
      headsep    = 1.9cm,
      footskip   = 1.45cm,
    }%
    \ifthu@output@print
      \geometry{
        left       = 3.2cm,
        right      = 3cm,
      }%
    \else
      \geometry{
        hmargin    = 3cm,
      }%
    \fi
  \else
    \geometry{
      margin     = 3cm,
      headheight = 2.7cm,
      headsep    = 0.3cm,
      footskip   = 0.8cm,
    }%
  \fi
}
\note@set@geometry
\note@option@hook{degree}{\note@set@geometry}
\note@option@hook{output}{\note@set@geometry}
%    \end{macrocode}

\subsection{主文档格式}
\label{sec:mainbody}

\subsubsection{Three matters}

% \begin{macro}{\cleardoublepage}
% 对于 \textsl{openright} 选项,必须保证章首页右开,且如果前章末页无内容须
% 清空其页眉页脚。
%    \begin{macrocode}
\def\cleardoublepage{%
  \clearpage
  \if@twoside
    \ifthu@output@print
      \ifodd\c@page
      \else
        \thispagestyle{empty}%
        \hbox{}%
        \newpage
        \if@twocolumn
          \hbox{}\newpage
        \fi
      \fi
    \fi
  \fi
}
%    \end{macrocode}
% \end{macro}
%
% \begin{macro}{\frontmatter}
% \begin{macro}{\mainmatter}
% \begin{macro}{\backmatter}
% 我们的单面和双面模式与常规的不太一样。
%    \begin{macrocode}
\renewcommand\frontmatter{%
  \cleardoublepage
  \@mainmatterfalse
  \pagenumbering{Roman}%
}
\renewcommand\mainmatter{%
  \cleardoublepage
  \@mainmattertrue
  \pagenumbering{arabic}%
}
\renewcommand\backmatter{%
  \if@openright
    \cleardoublepage
  \else
    \clearpage
  \fi
  \@mainmatterfalse
  \notesetup{toc-depth = 0}%
}
%    \end{macrocode}
% \end{macro}
% \end{macro}
% \end{macro}
%
% \subsubsection{字体}
% \label{sec:font}
% 使用 \pkg{fontspec} 配置字体。
%    \begin{macrocode}
\ifthu@fontset@mac
  \notesetup{
    font     = times,
    cjk-font = mac,
  }
\else
  \ifthu@fontset@windows
    \notesetup{
      font     = times,
      cjk-font = windows,
    }
  \else
    \ifthu@fontset@fandol
      \notesetup{
        font     = termes,
        cjk-font = fandol,
      }
    \else
      \ifthu@fontset@ubuntu
        \notesetup{
          font     = termes,
          cjk-font = noto,
        }
      \fi
    \fi
  \fi
\fi
%    \end{macrocode}
%
% 检测系统
%    \begin{macrocode}
\ifthu@system@auto
  \IfFileExists{/System/Library/Fonts/Menlo.ttc}{
    \notesetup{system = mac}
  }{
    \IfFileExists{/dev/null}{
      \IfFileExists{null:}{
        \notesetup{system = windows}
      }{
        \notesetup{system = unix}
      }
    }{
      \notesetup{system = windows}
    }
  }
\fi
%    \end{macrocode}
%
% XITS 字体。
% XITS 的文件名在 v1.109 2018-09-30
% 从 \file{xits-regular.otf}、\file{xits-math.otf} 分别改为
% \file{XITS-Regular.otf}、\file{XITSMath-Regular.otf}。
%    \begin{macrocode}
\let\note@font@family@xits\@empty
\newcommand\note@set@xits@names{%
  \ifx\note@font@family@xits\@empty
    \IfFontExistsTF{XITSMath-Regular.otf}{%
      \gdef\note@font@family@xits{XITS}%
      \gdef\note@font@style@xits@rm{Regular}%
      \gdef\note@font@style@xits@bf{Bold}%
      \gdef\note@font@style@xits@it{Italic}%
      \gdef\note@font@style@xits@bfit{BoldItalic}%
      \gdef\note@font@name@xits@math@rm{XITSMath-Regular}%
      \gdef\note@font@name@xits@math@bf{XITSMath-Bold}%
    }{%
      \gdef\note@font@family@xits{xits}%
      \gdef\note@font@style@xits@rm{regular}%
      \gdef\note@font@style@xits@bf{bold}%
      \gdef\note@font@style@xits@it{italic}%
      \gdef\note@font@style@xits@bfit{bolditalic}%
      \gdef\note@font@name@xits@math@rm{xits-math}%
      \gdef\note@font@name@xits@math@bf{xits-mathbold}%
    }%
  \fi
}
%    \end{macrocode}
%
% Libertinus 字体。
% Libertinus 的文件名在 v6.7 2019-04-03 从小写改为驼峰式,
% 在大小写敏感的平台上需要进行判断。
%    \begin{macrocode}
\let\note@font@family@libertinus\@empty
\newcommand\note@set@libertinus@names{%
  \ifx\note@font@family@libertinus\@empty
    \IfFontExistsTF{LibertinusSerif-Regular.otf}{%
      \gdef\note@font@family@libertinus@serif{LibertinusSerif}%
      \gdef\note@font@family@libertinus@sans{LibertinusSans}%
      \gdef\note@font@name@libertinus@math{LibertinusMath-Regular}%
      \gdef\note@font@style@libertinus@rm{Regular}%
      \gdef\note@font@style@libertinus@bf{Bold}%
      \gdef\note@font@style@libertinus@it{Italic}%
      \gdef\note@font@style@libertinus@bfit{BoldItalic}%
    }{%
      \gdef\note@font@family@libertinus@serif{libertinusserif}%
      \gdef\note@font@family@libertinus@sans{libertinussans}%
      \gdef\note@font@name@libertinus@math{libertinusmath-regular}%
      \gdef\note@font@style@libertinus@rm{regular}%
      \gdef\note@font@style@libertinus@bf{bold}%
      \gdef\note@font@style@libertinus@it{italic}%
      \gdef\note@font@style@libertinus@bfit{bolditalic}%
    }%
  \fi
}
%    \end{macrocode}
%
% STIX Two 字体。
% STIX 文件名在 v2.10 2020-12-19 从
% \file{STIX2Text-Regular.otf}、\file{STIX2Math.otf} 分别改为
% \file{STIXTwoText-Regular.otf}、\file{STIXTwoMath-Regular.otf}。
%    \begin{macrocode}
\let\note@font@family@stix\@empty
\newcommand\note@set@stix@names{%
  \ifx\note@font@family@stix\@empty
    \IfFontExistsTF{STIXTwoText-Regular.otf}{%
      \gdef\note@font@family@stix{STIXTwoText}%
      \gdef\note@font@name@stix@math{STIXTwoMath-Regular}%
    }{%
      \gdef\note@font@family@stix{STIX2Text}%
      \gdef\note@font@name@stix@math{STIX2Math}%
    }%
  \fi
}
%    \end{macrocode}
%
% 《撰写手册》要求西文字体使用 Times New Roman 和 Arial,
% 但是在 Linux 下没有这两个字体,所以使用它们的克隆版 TeX Gyre Termes 和
% TeX Gyre Heros。
%    \begin{macrocode}
\ifthu@font@auto
  \ifthu@system@unix
    \notesetup{font=termes}
  \else
    \notesetup{font=times}
  \fi
\fi
\newcommand\note@load@font@external{%
  \setmainfont[
    Path=fonts/,
    Extension = .ttf,
    UprightFont = * ,
    BoldFont = *bd ,
    ItalicFont = *i ,
    BoldItalicFont = *bi
  ]{times}%英文默认为Times New Roman
  \setsansfont[
    Path=fonts/,
    Extension = .ttf,
    UprightFont = * ,
    BoldFont = *bd ,
    ItalicFont = *i ,
    BoldItalicFont = *bi
  ]{arial}%
  \setmonofont[
    Path=fonts/,
    Extension = .ttf,
    UprightFont = * ,
    BoldFont = *bd ,
    ItalicFont = *i ,
    BoldItalicFont = *bi,
    Scale = MatchLowercase
  ]{cour}%
}
\newcommand\note@load@font@times{%
  \setmainfont{Times New Roman}%
  \setsansfont{Arial}%
  \setmonofont{Courier New}[Scale = MatchLowercase]%
}
\newcommand\note@load@font@termes{%
  \setmainfont{texgyretermes}[
    Extension      = .otf,
    UprightFont    = *-regular,
    BoldFont       = *-bold,
    ItalicFont     = *-italic,
    BoldItalicFont = *-bolditalic,
  ]%
  \note@load@texgyre@sans@mono
}
\newcommand\note@load@texgyre@sans@mono{%
  \setsansfont{texgyreheros}[
    Extension      = .otf,
    UprightFont    = *-regular,
    BoldFont       = *-bold,
    ItalicFont     = *-italic,
    BoldItalicFont = *-bolditalic,
  ]%
  \setmonofont{texgyrecursor}[
    Extension      = .otf,
    UprightFont    = *-regular,
    BoldFont       = *-bold,
    ItalicFont     = *-italic,
    BoldItalicFont = *-bolditalic,
    Ligatures      = CommonOff,
  ]%
}
\newcommand\note@load@font@xits{%
  \note@set@xits@names
  \setmainfont{\note@font@family@xits}[
    Extension      = .otf,
    UprightFont    = *-\note@font@style@xits@rm,
    BoldFont       = *-\note@font@style@xits@bf,
    ItalicFont     = *-\note@font@style@xits@it,
    BoldItalicFont = *-\note@font@style@xits@bfit,
  ]%
  \note@load@texgyre@sans@mono
}
\newcommand\note@load@font@libertinus{%
  \note@set@libertinus@names
  \setmainfont{\note@font@family@libertinus@serif}[
    Extension      = .otf,
    UprightFont    = *-\note@font@style@libertinus@rm,
    BoldFont       = *-\note@font@style@libertinus@bf,
    ItalicFont     = *-\note@font@style@libertinus@it,
    BoldItalicFont = *-\note@font@style@libertinus@bfit,
  ]%
  \setsansfont{\note@font@family@libertinus@sans}[
    Extension      = .otf,
    UprightFont    = *-\note@font@style@libertinus@rm,
    BoldFont       = *-\note@font@style@libertinus@bf,
    ItalicFont     = *-\note@font@style@libertinus@it,
  ]%
  \setmonofont{lmmonolt10}[
    Extension      = .otf,
    UprightFont    = *-regular,
    BoldFont       = *-bold,
    ItalicFont     = *-oblique,
    BoldItalicFont = *-boldoblique,
  ]%
}
\@namedef{thu@load@font@lm}{%
  \setmainfont{lmroman10}[
    Extension      = .otf,
    UprightFont    = *-regular,
    BoldFont       = *-bold,
    ItalicFont     = *-italic,
    BoldItalicFont = *-bolditalic,
  ]%
  \setsansfont{lmsans10}[
    Extension      = .otf,
    UprightFont    = *-regular,
    BoldFont       = *-bold,
    ItalicFont     = *-oblique,
    BoldItalicFont = *-boldoblique,
  ]%
  \setmonofont{lmmonolt10}[
    Extension      = .otf,
    UprightFont    = *-regular,
    BoldFont       = *-bold,
    ItalicFont     = *-oblique,
    BoldItalicFont = *-boldoblique,
  ]%
}
\newcommand\note@load@font{%
  \@nameuse{thu@load@font@\note@font}%
}
\note@load@font
\note@option@hook{font}{\note@load@font}
%    \end{macrocode}

%
% 使用 \pkg{unicode-math} 配置数学字体
%    \begin{macrocode}
\ifthu@language@chinese
  \notesetup{math-style=GB}%
\else
  \notesetup{math-style=TeX}%
\fi
\note@option@hook{language}{%
  \ifthu@language@chinese
    \notesetup{math-style=GB}%
  \else
    \notesetup{math-style=TeX}%
  \fi
}
\newcommand\note@xits@integral@stylistic@set{%
  %\ifthu@language@english
  \ifthu@math@style@GB
    8%upright
  \fi
}
\newcommand\note@load@math@font@xits{%
  \note@set@xits@names
  \setmathfont{\note@font@name@xits@math@rm}[
    Extension    = .otf,
    BoldFont     = \note@font@name@xits@math@bf,
    StylisticSet = \note@xits@integral@stylistic@set,
  ]%
  \setmathfont{\note@font@name@xits@math@rm}[
    Extension    = .otf,
    StylisticSet = 1,
    range        = {cal,bfcal},
  ]%
}
\newcommand\note@load@math@font@times{%
  \setmathfont{texgyretermes-math}[
    Extension    = .otf,
    StylisticSet = \note@xits@integral@stylistic@set,
  ]%
  \setmathfont{texgyretermes-math}[
    Extension    = .otf,
    StylisticSet = 1,
    range        = {cal,bfcal},
  ]%
}
\newcommand\note@load@math@font@stix{%
  \note@set@stix@names
  \setmathfont{\note@font@name@stix@math}[
    Extension    = .otf,
    StylisticSet = \note@xits@integral@stylistic@set,
  ]%
  \setmathfont{\note@font@name@stix@math}[
    Extension    = .otf,
    StylisticSet = 1,
    range        = {cal,bfcal},
  ]%
}
\newcommand\note@load@math@font@libertinus{%
  \note@set@libertinus@names
  \setmathfont{\note@font@name@libertinus@math .otf}%
}
\newcommand\note@load@math@font@lm{%
  \setmathfont{latinmodern-math.otf}%
}
\newcommand\note@load@math@font@cambria{%
  \setmathfont[Path = "fonts/"]{CambriaMath.otf}%
}
%    \end{macrocode}
%
% 中文字体
%    \begin{macrocode}
\ifthu@cjk@font@auto
  \ifthu@system@mac
    \notesetup{cjk-font = mac}
  \else
    \ifthu@system@windows
      \notesetup{cjk-font = windows}
    \else
      \IfFontExistsTF{Noto Serif CJK SC}{
        \notesetup{cjk-font = noto}
      }{
        \notesetup{cjk-font = fandol}
      }
    \fi
  \fi
\fi
\newcommand\note@load@cjk@font@external{%
  \xeCJKsetup{EmboldenFactor=3}%
  \setCJKmainfont[
    AutoFakeBold = true,
    Path = "fonts/",
    ItalicFont   = SimKai.ttf,
  ]{SimSun.ttc}%
  \setCJKsansfont[Path = "fonts/",AutoFakeBold]{SimHei.ttf}%
  \setCJKmonofont[Path = "fonts/",AutoFakeBold]{SimFang.ttf}%
  \setCJKfamilyfont{zhsong}[Path = "fonts/",AutoFakeBold]{SimSun.ttc}%
  \setCJKfamilyfont{zhhei}[Path = "fonts/",AutoFakeBold]{SimHei.ttf}%
  \setCJKfamilyfont{zhkai}[Path = "fonts/",AutoFakeBold]{SimKai.ttf}%
  \setCJKfamilyfont{zhfs}[Path = "fonts/",AutoFakeBold]{SimFang.ttf}%
}
\newcommand\note@load@cjk@font@windows{%
  \xeCJKsetup{EmboldenFactor=3}%
  \setCJKmainfont{SimSun}[
    AutoFakeBold = true,
    ItalicFont   = KaiTi,
  ]%
  \setCJKsansfont{SimHei}[AutoFakeBold]%
  \setCJKmonofont{FangSong}%
  \setCJKfamilyfont{zhsong}{SimSun}[AutoFakeBold]%
  \setCJKfamilyfont{zhhei}{SimHei}[AutoFakeBold]%
  \setCJKfamilyfont{zhkai}{KaiTi}%
  \setCJKfamilyfont{zhfs}{FangSong}%
}
\newcommand\note@load@cjk@font@mac{%
  \setCJKmainfont{Songti SC}[
    UprightFont    = * Light,
    BoldFont       = * Bold,
    ItalicFont     = Kaiti SC,
    BoldItalicFont = Kaiti SC Bold,
  ]%
  \setCJKsansfont{Heiti SC}[BoldFont=* Medium]%
  \setCJKmonofont{STFangsong}
  \setCJKfamilyfont{zhsong}{Songti SC}[
    UprightFont = * Light,
      BoldFont  = * Bold,
  ]%
  \setCJKfamilyfont{zhhei}{Heiti SC}[
    UprightFont = * Light,
    BoldFont    = * Medium,
  ]%
  \setCJKfamilyfont{zhfs}{STFangsong}%
  \setCJKfamilyfont{zhkai}{Kaiti SC}[BoldFont = * Bold]%
  \setCJKfamilyfont{zhli}{Baoli SC}%
  \setCJKfamilyfont{zhyuan}{Yuanyi SC}[
    UprightFont = * Light,
    BoldFont    = * Bold,
  ]%
}
%    \end{macrocode}
%
% 注意 Noto CJK 的 regular 字重名字不带“Regular”。
%    \begin{macrocode}
\newcommand\note@load@cjk@font@noto{%
  \setCJKmainfont{Noto Serif CJK SC}[
    UprightFont    = * Light,
    BoldFont       = * Bold,
    ItalicFont     = FandolKai-Regular,
    ItalicFeatures = {Extension = .otf},
  ]%
  \setCJKsansfont{Noto Sans CJK SC}[
    BoldFont    = * Medium,
  ]%
  \setCJKmonofont{Noto Sans Mono CJK SC}%
  \setCJKfamilyfont{zhsong}{Noto Serif CJK SC}[
    UprightFont = * Light,
    UprightFont = * Bold,
  ]%
  \setCJKfamilyfont{zhhei}{Noto Sans CJK SC}[
    BoldFont    = * Medium,
  ]%
  \setCJKfamilyfont{zhfs}{FandolFang}[
    Extension   = .otf,
    UprightFont = *-Regular,
  ]%
  \setCJKfamilyfont{zhkai}{FandolKai}[
    Extension   = .otf,
    UprightFont = *-Regular,
  ]%
}
\newcommand\note@load@cjk@font@fandol{%
  \setCJKmainfont{FandolSong}[
    Extension   = .otf,
    UprightFont = *-Regular,
    BoldFont    = *-Bold,
    ItalicFont  = FandolKai-Regular,
  ]%
  \setCJKsansfont{FandolHei}[
    Extension   = .otf,
    UprightFont = *-Regular,
    BoldFont    = *-Bold,
  ]%
  \setCJKmonofont{FandolFang}[
    Extension   = .otf,
    UprightFont = *-Regular,
  ]%
  \setCJKfamilyfont{zhsong}{FandolSong}[
    Extension   = .otf,
    UprightFont = *-Regular,
    BoldFont    = *-Bold,
  ]%
  \setCJKfamilyfont{zhhei}{FandolHei}[
    Extension   = .otf,
    UprightFont = *-Regular,
    BoldFont    = *-Bold,
  ]%
  \setCJKfamilyfont{zhfs}{FandolFang}[
    Extension   = .otf,
    UprightFont = *-Regular,
  ]%
  \setCJKfamilyfont{zhkai}{FandolKai}[
    Extension   = .otf,
    UprightFont = *-Regular,
  ]%
}
\ifthu@cjk@font@none\else
  \providecommand\songti{\CJKfamily{zhsong}}
  \providecommand\heiti{\CJKfamily{zhhei}}
  \providecommand\fangsong{\CJKfamily{zhfs}}
  \providecommand\kaishu{\CJKfamily{zhkai}}
\fi
\newcommand\note@load@cjk@font{%
  %\setCJKfamilyfont{FZXBSJ}[Path = "fonts/",AutoFakeBold]{FZXBSJ.otf}%
  %\setCJKfamilyfont{SourceHanSansCN}[Path = "fonts/"]{SourceHanSansCN-Regular.otf}%
  %\setCJKfamilyfont{SourceHanSerifCN}[Path = "fonts/"]{SourceHanSerifCN-Regular.otf}%
  %\newfontfamily\siyuanheiti@en[Path = "fonts/"]{SourceHanSansCN-Regular.otf}%
  %\newfontfamily\siyuansongti@en[Path = "fonts/"]{SourceHanSerifCN-Regular.otf}%
  \@nameuse{thu@load@cjk@font@\note@cjk@font}%
  %\providecommand{\fzxbiaosong}{\CJKfamily{FZXBSJ}}
  %\providecommand{\siyuanheiti}{\CJKfamily{SourceHanSansCN}}
  %\providecommand{\siyuansongti}{\CJKfamily{SourceHanSerifCN}}
}
\note@load@cjk@font
\note@option@hook{cjk-font}{\note@load@cjk@font}
%    \end{macrocode}
%
% \begin{macro}{\normalsize}
% 正文小四号 (12bp) 字,行距为固定值 20 bp。
% 其他字号的行距按照相同的比例设置。
%
% 表达式行的行距为单倍行距,段前空 6 磅,段后空 6 磅。
%    \begin{macrocode}
\renewcommand\normalsize{%
  \@setfontsize\normalsize{12bp}{20bp}%
  \abovedisplayskip 6bp%
  \abovedisplayshortskip 6bp%
  \belowdisplayshortskip 6bp%
  \belowdisplayskip \abovedisplayskip
}
\normalsize
\ifx\MakeRobust\@undefined \else
    \MakeRobust\normalsize
\fi
\DeclareRobustCommand\small{%
   \@setfontsize\small{10.5bp}{17.5bp}%
   \abovedisplayskip 6bp%
   \abovedisplayshortskip 6bp%
   \belowdisplayshortskip 6bp%
   \def\@listi{\leftmargin\leftmargini
               \topsep \z@skip
               \parsep \z@skip
               \itemsep \z@skip}%
   \belowdisplayskip \abovedisplayskip
}
\DeclareRobustCommand\footnotesize{%
   \@setfontsize\footnotesize{9bp}{15bp}%
   \abovedisplayskip 6bp%
   \abovedisplayshortskip 6bp%
   \belowdisplayshortskip 6bp%
   \def\@listi{\leftmargin\leftmargini
               \topsep \z@skip
               \parsep \z@skip
               \itemsep \z@skip}%
   \belowdisplayskip \abovedisplayskip
}
\DeclareRobustCommand\scriptsize{\@setfontsize\scriptsize{7.5bp}{12.5bp}}
\DeclareRobustCommand\tiny{\@setfontsize\tiny{6.5bp}{10.83bp}}
\DeclareRobustCommand\large{\@setfontsize\large{15bp}{25bp}}
\DeclareRobustCommand\Large{\@setfontsize\Large{18bp}{30bp}}
\DeclareRobustCommand\LARGE{\@setfontsize\LARGE{22bp}{36.67bp}}
\DeclareRobustCommand\huge{\@setfontsize\huge{24bp}{40bp}}
\DeclareRobustCommand\Huge{\@setfontsize\Huge{26bp}{43.33bp}}
%    \end{macrocode}
% \end{macro}


% WORD 中的字号对应该关系如下(1bp = 72.27/72 pt):
% \begin{center}
% \begin{longtable}{llll}
% \toprule
% 初号 & 42bp & 14.82mm & 42.1575pt \\
% 小初 & 36bp & 12.70mm & 36.135 pt \\
% 一号 & 26bp & 9.17mm & 26.0975pt \\
% 小一 & 24bp & 8.47mm & 24.09pt \\
% 二号 & 22bp & 7.76mm & 22.0825pt \\
% 小二 & 18bp & 6.35mm & 18.0675pt \\
% 三号 & 16bp & 5.64mm & 16.06pt \\
% 小三 & 15bp & 5.29mm & 15.05625pt \\
% 四号 & 14bp & 4.94mm & 14.0525pt \\
% 小四 & 12bp & 4.23mm & 12.045pt \\
% 五号 & 10.5bp & 3.70mm & 10.59375pt \\
% 小五 & 9bp & 3.18mm & 9.03375pt \\
% 六号 & 7.5bp & 2.56mm & \\
% 小六 & 6.5bp & 2.29mm & \\
% 七号 & 5.5bp & 1.94mm & \\
% 八号 & 5bp & 1.76mm & \\\bottomrule
% \end{longtable}
% \end{center}
%
% \begin{macro}{\note@def@fontsize}
% 根据习惯定义字号。用法:
%
% \cs{thu@def@fontsize}\marg{字号名称}\marg{磅数}
%
% 避免了字号选择和行距的紧耦合。所有字号定义时为单倍行距,并提供选项指定行距倍数。
%    \begin{macrocode}
\def\note@def@fontsize#1#2{%
  \expandafter\newcommand\csname #1\endcsname[1][1.3]{%
    \fontsize{#2}{##1\dimexpr #2}\selectfont}}
%    \end{macrocode}
% \end{macro}
%
% 一组字号定义。
%    \begin{macrocode}
\note@def@fontsize{chuhao}{42bp}
\note@def@fontsize{xiaochu}{36bp}
\note@def@fontsize{yihao}{26bp}
\note@def@fontsize{xiaoyi}{24bp}
\note@def@fontsize{erhao}{22bp}
\note@def@fontsize{xiaoer}{18bp}
\note@def@fontsize{sanhao}{16bp}
\note@def@fontsize{xiaosan}{15bp}
\note@def@fontsize{sihao}{14bp}
\note@def@fontsize{xiaosi}{12bp}
\note@def@fontsize{wuhao}{10.5bp}
\note@def@fontsize{xiaowu}{9bp}
\note@def@fontsize{liuhao}{7.5bp}
\note@def@fontsize{xiaoliu}{6.5bp}
\note@def@fontsize{qihao}{5.5bp}
\note@def@fontsize{bahao}{5bp}
%    \end{macrocode}


% \subsubsection{语言设置}
%
% 定义 \cs{thu@main@language},当在导言区修改 \option{language} 时,
% 保存为论文的主要语言;
% \cs{thu@reset@main@language} 则用于正文中恢复为主要语言。
%    \begin{macrocode}
\notesetup{main-language=\note@language}%
\let\note@main@language\note@language
\note@option@hook{language}{%
  \ifx\@begindocumenthook\@undefined\else
    \notesetup{main-language=\note@language}%
    \let\note@main@language\note@language
  \fi
}
\newcommand\note@reset@main@language{%
    \notesetup{language = \note@main@language}%
    \let\note@language\note@main@language
}
%    \end{macrocode}
%
% 根据语言设置各章节的名称,只有在导言区设置 \option{degree} 和
% \option{language} 时会修改,而在正文局部切换语言时则不变。
%    \begin{macrocode}
\newcommand\note@set@chapter@names{%
  \ifthu@main@language@chinese
    \def\bibname{参考文献}%
    \def\appendixname{附录}%
    \def\indexname{索引}%
    \def\note@acknowledgements@name{致\quad 谢}%
    \def\note@conclusion@name{结\quad 论}%
    \ifthu@degree@bachelor
      \def\contentsname{目\quad 录}%
      \def\listfigurename{插图索引}%
      \def\listtablename{表格索引}%
      \def\note@list@figure@table@name{插图和附表索引}%
      \def\note@list@algorithm@name{算法索引}%
      \def\listequationname{公式索引}%
      \def\note@denotation@name{主要符号对照表}%
      \def\note@resume@name{在学期间参加课题的研究成果}%
    \else
      \def\listfigurename{插图清单}%
      \def\listtablename{附表清单}%
      \def\note@list@figure@table@name{插图和附表清单}%
      \def\note@list@algorithm@name{算法清单}%
      \def\listequationname{公式清单}%
      \ifthu@degree@graduate
        \def\contentsname{目\quad 录}%
        \def\note@denotation@name{符号和缩略语说明}%
        \def\note@resume@name{个人简历、在学期间完成的相关学术成果}%
        \def\note@statement@name{南方科技大学学位论文原创性声明和使用授权说明}%
        \def\note@statement@toc@name{南方科技大学学位论文原创性声明和使用授权说明}%
      \else  % degree = postdoc
        \def\contentsname{目\qquad 次}%
        \def\note@denotation@name{符号表}%
        \def\note@resume@name{个人简历、发表的学术论文与科研成果}%
      \fi
    \fi
  \else
    \ifthu@main@language@english
      \def\indexname{Index}%
      \ifthu@degree@bachelor
        \def\contentsname{CONTENTS}%
        \def\listfigurename{FIGURES}%
        \def\listtablename{TABLES}%
        \def\note@list@figure@table@name{FIGURES AND TABLES}%
        \def\note@list@algorithm@name{ALGORITHMS}%
        \def\listequationname{EQUATIONS}%
        \def\note@denotation@name{ABBREVIATIONS}%
        \def\bibname{REFERENCES}%
        \def\appendixname{APPENDIX}%
        \def\note@acknowledgements@name{ACKNOWLEDGEMENTS}%
        \def\note@resume@name{PUBLICATIONS}%
      \else
        \def\contentsname{TABLE OF CONTENTS}%
        \def\listfigurename{LIST OF FIGURES}%
        \def\listtablename{LIST OF TABLES}%
        \def\note@list@figure@table@name{LIST OF FIGURES AND TABLES}%
        \def\note@list@algorithm@name{LIST OF ALGORITHMS}%
        \def\listequationname{LIST OF EQUATIONS}%
        \def\note@denotation@name{LIST OF SYMBOLS AND ACRONYMS}%
        \def\note@conclusion@name{CONCLUSION}%
        \def\bibname{REFERENCES}%
        \def\appendixname{APPENDIX}%
        \def\note@acknowledgements@name{ACKNOWLEDGEMENTS}%
        \def\note@resume@name{RESUME AND ACADEMIC ACHIEVEMENTS}%
        \def\note@statement@name{DECLARATION OF ORIGINALITY AND AUTHORIZATION OF THESIS}%
        \def\note@statement@toc@name{DECLARATION OF ORIGINALITY AND AUTHORIZATION OF THESIS, SOUTHERN UNIVERSITY OF SCIENCE AND TECHNOLOGY}%
      \fi
    \fi
  \fi
}
\note@set@chapter@names
\note@option@hook{degree}{\note@set@chapter@names}
\note@option@hook{main-language}{\note@set@chapter@names}
%    \end{macrocode}
%
% 这部分名称在正文中局部地修改语言时会发生变化,比如英文摘要、
% 本科生附录的阅读报告。
%    \begin{macrocode}
\newcommand\note@set@names{%
  \ifthu@language@chinese
    \ctexset{
      figurename = 图,
      tablename  = 表,
    }%
    \def\note@algorithm@name{算法}%
    \def\note@equation@name{公式}%
    \def\note@assumption@name{假设}%
    \def\note@definition@name{定义}%
    \def\note@proposition@name{命题}%
    \def\note@lemma@name{引理}%
    \def\note@theorem@name{定理}%
    \def\note@axiom@name{公理}%
    \def\note@corollary@name{推论}%
    \def\note@exercise@name{练习}%
    \def\note@example@name{例}%
    \def\note@remark@name{注释}%
    \def\note@problem@name{问题}%
    \def\note@conjecture@name{猜想}%
    \def\note@proof@name{证明}%
    \def\note@theorem@separator{:}%
  \else
    \ifthu@language@english
      \ctexset{
        figurename = {Figure},
        tablename  = {Table},
      }%
      \def\note@algorithm@name{Algorithm}%
      \def\note@equation@name{Equation}% 仅在listofequations中使用
      % 以下环境均为正文中使用的定理环境
      \def\note@assumption@name{Assumption}%
      \def\note@definition@name{Definition}%
      \def\note@proposition@name{Proposition}%
      \def\note@lemma@name{Lemma}%
      \def\note@theorem@name{Theorem}%
      \def\note@axiom@name{Axiom}%
      \def\note@corollary@name{Corollary}%
      \def\note@exercise@name{Exercise}%
      \def\note@example@name{Example}%
      \def\note@remark@name{Remark}%
      \def\note@problem@name{Problem}%
      \def\note@conjecture@name{Conjecture}%
      \def\note@proof@name{Proof}%
      \def\note@theorem@separator{:}%
    \fi
  \fi
}
\note@set@names
\note@option@hook{language}{\note@set@names}
%    \end{macrocode}
%
% 带圈数字和星号使用中文字体。
%    \begin{macrocode}
\xeCJKDeclareCharClass{CJK}{"2460 -> "2473}
\xeCJKDeclareCharClass{CJK}{"2605}
%    \end{macrocode}
%
% \newcommand\unicodechar[1]{U#1(\symbol{"#1})}
% 由于 Unicode 的一些标点符号是中西文混用的:
% \unicodechar{00B7}、
% \unicodechar{2013}、
% \unicodechar{2014}、
% \unicodechar{2018}、
% \unicodechar{2019}、
% \unicodechar{201C}、
% \unicodechar{201D}、
% \unicodechar{2025}、
% \unicodechar{2026}、
% \unicodechar{2E3A},
% 所以要根据语言设置正确的字体。
% \footnote{\url{https://github.com/CTeX-org/ctex-kit/issues/389}}
% 此外切换语言时,有一部分名称是需要被重新定义的。
%    \begin{macrocode}
\newcommand\note@set@punctuations{%
  \ifthu@language@chinese
    \xeCJKDeclareCharClass{FullLeft}{"2018, "201C}%
    \xeCJKDeclareCharClass{FullRight}{
      "00B7, "2019, "201D, "2013, "2014, "2025, "2026, "2E3A,
    }%
  \else
    \ifthu@language@english
      \xeCJKDeclareCharClass{HalfLeft}{"2018, "201C}%
      \xeCJKDeclareCharClass{HalfRight}{
        "00B7, "2019, "201D, "2013, "2014, "2025, "2026, "2E3A,
      }%
    \fi
  \fi
}
\note@set@punctuations
\note@option@hook{language}{\note@set@punctuations}
%    \end{macrocode}
%

\subsubsection{页眉页脚}
\label{sec:headerfooter}

% \pkg{fancyhdr} 定义页眉页脚很方便,但是有一个非常隐蔽的坑。
% 第一次调用 \pkg{fancyhdr} 定义的样式时会修改 \cs{chaptermark},
% 这会导致页眉信息错误(多余章号并且英文大写)。
% 这是因为在 \cs{ps@fancy} 中对 \cs{chaptermark} 进行重定义,
% 所以我们先调用 \cs{ps@fancy},再修改 \cs{chaptermark}。

\begin{macrocode}
	\pagestyle{fancy}	
\end{macrocode}


% 定义页眉和页脚。
% 研究生要求:
% 页眉宋体五号字,宋体五号字居中书写;
% 页码五号 Times New Roman 体。
%
% 本科生要求:
% 页眉:无;
% 页码:位于页面底端,居中书写。
%
% 本科外文专业要求页码字号 12pt。

\begin{macrocode}
	\fancypagestyle{plain}{
		\fancyhf{}
		\renewcommand\footrulewidth{0pt}
		\ifnote@degree@bachelor
			\renewcommand\headrulewidth{0pt}
			\fancyfoot[C]{
				\ifnote@main@language@chinese 
					\xiaowu 
				\else
					\normalsize
				\fi
				\thepage
			}
			\let\@mkboth\@goobletwo
			\let\chaptermark\@gobble
		\else
			\renewcommand\headrulewidth{0.75bp}
		    \fancyhead[C]{\wuhao\leftmark}%
			\fancyfoot[C]{\wuhao\thepage}%
			\let\@mkboth\markboth
			\ifthu@main@language@chinese
				\def\chaptermark##1{%
					\markboth{\CTEXifname{\CTEXthechapter\quad}{}##1}{}%
				}%
			\else
				\def\chaptermark##1{%
					\markboth{\CTEXifname{\MakeTextUppercase{\CTEXthechapter\quad}}{}##1}{}%
				}%
			\fi
		\fi
		\let\sectionmark\@gobble
	}
\pagestyle{plain}

\end{macrocode}


