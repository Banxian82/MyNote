\section{分析化学}

\begin{itemize}
	\item 什么是分析化学: 测定物质的组成、含量、结构的一门学科。
	\item 它的任务是什么

	\begin{itemize}
		\item 测定组成:定性分析
		\item 测定含量:定量分析
		\item 测定结构:结构分析	
	\end{itemize}

	\item 作用: 
	
\end{itemize}

分析方法的分类及选择原则

分析方法分类
\begin{itemize}
	\item 依据分析工作的任务不同分为
	\begin{itemize}
		\item 定性分析
		\item 定量分析
		\item 结构分析
	\end{itemize}
		
	\item 分为
	\begin{itemize}
		\item 化学分析法
		\begin{itemize}
			\item 容量分析方法 (滴定分析法 (包括酸碱滴定法、氧化还原滴定法、沉淀滴定法、络合滴定法(也称配位滴定法))) 
			\item 重量分析方法 (缺点:麻烦)
				
		\end{itemize}
		\item 仪器分析法 (依据物质的性质和某一个组分建立的关系) (测定物质含量低的测定方法)

	\end{itemize}
\end{itemize}

\begin{definition}{定性分析 qualitative analysis}
	定性分析的任务是鉴定物质由哪些元素、原子团或化合物组成。
	
\end{definition}

\begin{definition}{定量分析 quantitative analysis}
	定量分析的任务是测定物质中有关成分的含量。
	
\end{definition}

\begin{definition}{结构分析 structure analysis}
	结构分析是研究物质的分子结构、晶体结构或综合形态。
	
\end{definition}


\section{分析过程及分析结果表示}
\subsection{分析过程}
取样 -> 预处理() -> 分析测试 -> 得到分析数据 -> 数据处理 -> 确定质量(从精密度和准确度看出质量好坏) -> 报告

\subsection{分析结果表示}
\begin{enumerate}
	\item 化学形式
	\begin{enumerate}
		\item 元素符号。例如:N、P.
		\item 实际存在形式 $cl^{-1}$
		\item 氧化物形式 例如 $Fe_2O_3$ 
		
	\end{enumerate}

	\item 含量形式 (的表示方式)
	\begin{enumerate}
		\item 固体试样 质量分数
		\item 液体试样 物质的量浓度(摩尔浓度 mol/L) 、质量摩尔浓度(mol/g)、质量浓度 (mg/L)、质量分数、体积分数
		\item 气体试样 质量浓度 (mg/L)、体积分数 ()
		
	\end{enumerate}
	
\end{enumerate}

\subsection{滴定实验}
滴定分析法是将一种已知准确浓度的试剂 (即标准溶液) 滴加到被测物质的溶液中,或将被测物质的溶液滴加到标准溶液中,直至所加的试剂与被测物质按化学式计量关系定量反应位置,然后根据试剂溶液的浓度和用量,计算被测物质的含量。


滴定分析法术语
\begin{itemize}
	\item 滴定反应
	\item 滴定指示剂
	\item 标准溶液: 已知准确浓度的试剂
	\item 滴定: 将滴定剂从滴定管中加入被测物质溶液中的\textbf{过程}
	\item 化学计量点: 加入的标准溶液与被测物质定量反应完全时,反应即到达了\textbf{化学计量点}。 
	\item 滴定剂
	\item 滴定终点: 在滴定过程中,指示剂改变颜色的那一点称为\textbf{滴定终点}(例如:滴定达到指示剂 (酚酞) 变色)
	\item 滴定误差: 滴定终点与化学计量点不同时,他们之间的分析误差称为\textbf{重点误差 (用$E_t$ 表示)} ( |$E_t$| <= 0.1\% )
	
\end{itemize}


滴定反应对化学反应的要求
\begin{enumerate}
	\item 有确定的化学计量关系 (即反应按照一定的反应方程式进行)
	\item 有较快的反应速率。(若反应速率慢,可采用加热或加入催化剂的方式加速反应进行)
	\item 反应定量进行
	\item 有合适的方法指示滴定终点。
	
\end{enumerate}

滴定反应对化学反应采用的滴定方式 

假设滴定剂为A,被测溶液为B,其他溶液为C。
\begin{enumerate}
	\item 直接滴定方式 : (A 滴 B) 滴定剂和被测物质直接反应

	\item 间接滴定方式 : (A 滴 (B+C) ) 被测溶液中加入另外一种物质,使其反应生成物质A,然后再使用滴定剂与物质A进行反应。(适用于待测物质不能与滴定剂直接起反应时)
	例如: 滴定剂为 $KM_2O_4$,被测溶液为:含$Ca^{2+}$,我们需要测定被测溶液中$Ca^{2+}$的含量,我们加入草酸 ($H_2C_2O_4$),使得草酸与钙离子发生反应得到草酸钙,然后使用高锰酸钾滴定被测溶液中的草酸根离子 ($C_2O_4^{2-}$) 反应,从而间接得知钙离子的含量。

	\item 返滴定方式: (C 滴 (A + B)) (适用于待测物质与滴定剂反应很慢或反应不能立即完成时。) 

	\item 置换滴定方式 : (A 滴 (B + C)) (适用于待测组分不一定按照反应式进行或产生副反应时)
	
\end{enumerate}

\subsection{标准溶液和基准物质}

\subsubsection{基准物质}
基准物质:可以用于直接配制标准溶液的物质。

基准物质应该满足下列要求:
\begin{enumerate}
	
	
\end{enumerate}

常用的基准物质:
\begin{itemize}
	\item 
	
\end{itemize}

\subsubsection{标准溶液}

标准溶液的配置方法:

\begin{enumerate}
	\item 直接法配置。准确称量一定量的基准物质,将其溶解后配置成一定体积的溶液,根据物质质量和溶液体积,即可计算出标准溶液的准确浓度基准物质 -> 标准溶液 (计算出量后称量)

		设基准物质B的摩尔质量为$M_B(g \dot mol^{-1})。质量为$m_B(g)$,则物质B的物质的量为
	\begin{equation*}
		n_B = m_B/M_B
	\end{equation*}
		若将其配制成体积为$V_B(L)$的标准溶液,它的浓度为
	\begin{equation*}
		c_B = \frac{n_B}{V_B} = \frac{m_B}{V_B M_B}		
	\end{equation*}

	\item 标定法 (也叫间接配置法): 用于某些物质不能直接配制成标准溶液,但可以将其配制成近似于所需浓度的溶液,然后用基准物质 (或已经用基准物质标定过的标准溶液)来标定它的准确浓度。
	\begin{enumerate}
		\item 粗配

		
	\end{enumerate}
	
\end{enumerate}

\subsection{滴定分析计算}

\begin{example}
	欲配置 $0.1000 mol\dot L^{-1} $ 的 $Na_{2}CO_{3}$ 标准溶液 500mL,问应称取基准物质 $Na_2CO_3$ 多少克?
	
	\[M_{Na_2CO_3} = 2 \times 22.99 + 1 \times 12.01 + 3 \times 16 = 105.99 \approx 106.00 g \dot mol^{-1}	\]
	\begin{align}
		
		m_{Na_2CO_3} & = c_{Na_2CO_3}V_{Na_2CO_3}M_{Na_2CO_3} \\ 
		& = 0.1000 mol \dot L^{-1} \times 0.5000 L \times 106.00 g\dot mol^{-1}
		& = 5.300 g
	\end{align}
\end{example}

\begin{example}
	有 0.1035 mol \dot L^{-1} 的 NaOH 标准溶液 500mL , 欲使其浓度恰好为 0.1000 mol \dot L^{-1}。需要加水多少毫升
	\begin{align}
		& 0.1035 mol \dot L^{-1} \times 0.5000 L = (0.5000mL + V_{水}) \times 0.1000 mol \dot L^{-1} \\ 
		& V_{水} = \frac{(0.1035 - 0.1000) \times 0.5000mL}{0.1000mol \dot L^{-1}}
		
	\end{align}
	
\end{example}


\begin{example}
	为标定 HCl 溶液,称取硼砂(Na_2B_4O_7 \dot H_2O) 0.471g, 用 HCl 溶液测定至化学计量点时消耗 24.20 mL。求 HCl 溶液的浓度。

	\begin{align}
		5H_2O + Na_2B_4O_7 + 2HCl = 2NaCl + 4H_3BO_3 \\ 
		& n_{HCl} = 2n_{Na_2B_4O_7} \\  
		& c_{HCl} = \frac{2m_{Na_2B_4O_7 \dot 10H_2O}}{M_{Na_2B_4O_7 \dot 10H_2O} V_{HCl}} \\ 

		
	\end{align}
	
\end{example}

\begin{example}
	计算 0.01500 mol \dot L^{-1} K——2Cr_2O_7 溶液对 Fe 和 Fe_3O_4 的滴定度

	
\end{example}

\begin{question}
	浓HCL, $w = 36 \%, \rho = 1.18 g/mL ,求盐酸的物质的量浓度$ 

	$c =   
	
\end{question}
\subsubsection{标准溶液表示方法}


\begin{enumerate}
	\item 基本单元。 酸碱反应以能够提供一个氢离子或结合一个氢氧根离子为一个基本单元,氧化还原反应以提供一个电子或得到一个电子为一个基本单元。 
	\item 滴定度  
\end{enumerate}



